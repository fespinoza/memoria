\chapter{Funcionamiento de la Solución}
\label{chap:funcionamiento_solucion}

A continuación, se mostrarán las características principales de la aplicación desarrollada, a modo de referencia para usuarios, mostrando pantallazos de la interfaz, pero además discutiendo cómo esta interfaz y estas características resuelven los problemas planteados y cumplen los requisitos definidos en la sección~\ref{sub:requisitos_funcionales}.

% 1. qué feature estoy describiendo?
% 2. como funciona? (en términos de uso de la aplicación)
% 3. qué requisito cumple este feature? (hablando de ventajas de este enfoqu)

\section{Registro e Ingreso de Usuarios} % (fold)
\label{sec:registro_e_ingreso_de_usuarios}

% Con el fin de mantener la privacidad de las redes sociales que sus usuarios crean, se necesita un sistema de autentificación de usuarios, el cual en este caso requiere una combinación de email y password, y que además requiere que el usuario se registre 

% section registro_e_ingreso_de_usuarios (end)

% ### 5.1 registro e ingreso de usuarios [screenshot]x2
% ### 5.2 creación de redes sociales [screenshot]

\section{Edición de Redes Sociales} % (fold)
\label{sec:edicion_de_redes_sociales}

Una vez el usuario tiene su cuenta y creo una red social, este está listo para ir agregando los datos relevantes a sus redes sociales, haciendo click sobre el link con el nombre de la red social recién creada

% TODO FIGURA MENÚ DE REDES SOCIALES

\subsection{Área de Edición de Redes Sociales} % (fold)
\label{sub:area_de_edicion_de_redes_sociales}

El área de edición de redes sociales consiste básicamente de 4 elementos: el canvas donde se crea el grafo de la red social (1), una barra de herramientas para la edición (2), un área donde se definen las familias de actores y relaciones (3) y un formulario con los detalles de las entidades seleccionadas (4).

% TODO FIGURA VISIÓN EDICIÓN GENERAL AQUÍ

Esta interfaz está pensada para tener la menor cantidad de elementos posibles, haciendo énfasis en las herramientas primordiales necesarias para la edición del grafo.

\subsubsection{Modos de Edición} % (fold)
\label{ssub:modos_de_edicion}

Para la zona de edición se cuenta con 4 modos de edición, de acuerdo a los cuales puedo realizar acciones diversas cuando hago click dentro de la zona del canvas.

% TODO FIGURA MODOS DE EDICIÓN

Los 4 modos existentes son:

  \begin{enumerate}
    \item \textbf{Herramienta de Movimiento}: esta herramienta me permite cambiar la posición de los nodos dentro del canvas a voluntad, además de al hacer click en estos seleccionarlos para la edición de sus detalles.
    \item \textbf{Modo Actor}: en este modo, al hacer click en algún punto del canvas creará un nuevo actor dentro de esta en la posición donde se especificó haciendo click, el enfoque será puesto automáticamente en el formulario de detalles del nuevo actor para rellenar rápidamente sus campos necesarios y confirmar la creación del nuevo actor.
    \item \textbf{Modo Relación}: en este modo, al hacer click en algún punto del canvas creará una nueva relación situada en esas coordenadas, el enfoque será puesto automáticamente en el formulario de edición de la relación, pero a diferencia de los actores, las relaciones son automáticamente guardadas al momento de ser creadas y cambian al \emph{modo Rol}.
    \item \textbf{Modo Rol}: en este modo, al momento de hacer click en un actor (manteniendo el botón presionado), puedo arrastrar una flecha y situarla sobre una relación, donde al soltar el botón del mouse, creará automáticamente un nuevo rol del actor seleccionado en la relación seleccionada, para después editar sus detalles en el formulario de roles correspondiente. Además, puedo repetir esta acción cuantas veces sea necesario.
  \end{enumerate}

% subsubsection modos_de_edición (end)

% subsection área_de_edición_de_redes_sociales (end)

\subsection{Creación y Edición de Familias} % (fold)
\label{sub:creacion_y_edicion_de_familias}

Para poder agrupar los nodos (actores y relaciones) dentro de familias, hay un área reservada para la creación propia de estas familias dentro de la red, en donde a continacíon se explican las principales operaciones con familias dentro de una red social.\\

% TODO FOTO AREA DE FAMILIAS

Para crear una familia se presiona el botón \emph{Add} en la sección de familias, con lo que aparece un formulario con el siguiente:

% TODO FORMULARIO DE EDICIÓN de FAMILIAS

Acá se rellena el nombre y se selecciona el color para mostrar los nodos de esta familia y el tipo de nodos a los cuáles se les asignará esta familia.\\

Una vez creada, se muestra la familia en el listado con un ícono de \emph{A} para el caso de familias de actores y de \emph{R} en el caso de familias de relaciones. Donde puedo editar sus detalles o eliminarlas con los botones que se encuentran a su lado.

% TODO FOTO DETALLE DE FAMILIA CREADA Y SUS OPERACIONES

Es importante destacar, que se puede presionar cualquier tipo de familia, en donde al hacer esto, se cambiará al \emph{modo Actor} o \emph{modo Relación} según corresponda y a continuación cuando creo un actor o relación, por defecto pertenecerá a la familia seleccionada.

% subsection creación_y_edición_de_familias (end)

\subsection{Creación y Edición de Actores} % (fold)
\label{sub:creacion_y_edicion_de_actores}

Para crear actores, se debe definir el modo de edición a \emph{modo Actor}~\ref{ssub:modos_de_edicion}, y hacer click en el canvas donde aparecerá un actor en dicho punto y se enfocará automáticamente el formulario de creación del actor.\\

% TODO INSERT FORMULARIO DE EDICION

En este formulario se puede rellenar el nombre (opcionalmente), seleccionar las familias a las que pertenece el actor para finalmente confirmar la creación, luego de esto, la información visual del actor es actualizada, mostrando al actor del color de la(s) familia(s) a la cual pertenece, además de un borde indicando de que es dicho actor el que está seleccionado en este momento.\\

% TODO ACTOR CON 2 FAMILIAS CREADO EN EL CANVAS

El actor puede ser editado en cualquier momento vía el formulario de actor, luego se presiona el botón de actualizar para persistir los cambios, o puede ser eliminado con el botón de borrar, después de confirmar en el cuadro de dialogo que aparece.

% TODO DIALOGO ELIMINACIÓN ACTOR

% subsection creación_y_edición_de_actores (end)

\subsection{Creación y Edición de Relaciones} % (fold)
\label{sub:creacion_y_edicion_de_relaciones}

Para crear una relación, se debe seleccionar el \emph{modo Relación}~\ref{ssub:modos_de_edicion}, posteriormente hacer click dentro del canvas en donde aparecerá la nueva relación y se mostrará el formulario de edición de la relación. A diferencia de los actores, las relaciones son creadas automáticamente, lo cual cambiará al modo de edición de roles, para agregar los roles correspondientes a la relación sin perder el contexto.

% TODO FORMULARIO DE EDICIÓN DE RELACIONES

% TODO check this!!! (relaciones-n-familias)
Las relaciones pueden o no tener un nombre, además de pertenecer a una familia, lo que generalmente denota el tipo de relación con la cual se está trabajando, ej: estudiaEn, dueñoDe, etc.\\

La relación puede ser editada en cualquier momento vía el formulario y presionando el botón de actualizar, o eliminada con el botón de borrar, luego de confirmar el cuadro de dialogo que aparece.

% TODO DIALOGO DE ELIMINACIÓN RELACIÓN

% subsection creación_y_edición_de_relaciones (end)

\subsection{Creación y Edición de Atributos en Nodos} % (fold)
\label{sub:creacion_y_edicion_de_atributos_en_nodos}

Una vez teniendo actores y relaciones creados dentro de la red social, es posible agregarles todos los atributos que se estimen conveniente por medio del formulario de edición de actores o relaciones, para esto, en la subsección de atributos en dicho formulario se puede agregar uno presionando el botón \emph{Add}, en donde puedo ingresar un atributo como un par key-value, por ejemplo puedo agregar el atributo \emph{Edad} (key) con el valor \emph{24} a un actor.

% TODO INSERCIÓN DE ATRIBUTOS

Para editar atributos, se pueden editar directamente en sus campos y luego presionar el botón \emph{Update} para que los cambios sean persistidos, o borrar un atributo presionando el ícono junto a la definición del mismo.

% subsection creación_y_edición_de_atributos_en_nodos (end)

\subsection{Creación y Edición de Roles} % (fold)
\label{sub:creacion_y_edicion_de_roles}

Los roles representan la participación de un actor en una relación, dicha participación o rol, puede tener un nombre o no, un ejemplo del último caso: en una relación de amistad entre 2 personas, puede haber un tercer actor que fue quien los introdujo, pero nuevamente, el nombre de un rol es opcional. Para crear un rol, se debe seleccionar el modo de edición de roles~\ref{ssub:modos_de_edicion}, luego con el mouse, pincho un actor y arrastro el mouse hacia una relación, al soltar el mouse el rol va a ser creado inmediatamente y el foco va a ser puesto dentro del formulario de edición del rol.

% TODO FORMULARIO EDICIÓN ROL

En este formulario puedo actualizar el nombre del rol o de ser necesario eliminar el rol. Es importante mencionar que los roles sólo serán creados desde un \emph{Actor} hacia una \emph{Relación}, cualquier otra combinación no resultará en la creación de un rol.

% subsection creación_y_edición_de_roles (end)

% section edición_de_redes_sociales (end)

% ### 5.4 exportación rdf [extracto de código]
% 
% * más vocabulario
% 
% ### 5.5 importación rdf
% ### 5.6 unión de redes sociales [screenshot]
% ### 5.7 Casos de estudio
% 
% Algún ejemplo concreto de una red un poco más grande creada con esta aplicación
