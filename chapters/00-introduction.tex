\begin{intro}
  % -- Contexto y littlesis --
  Existe una red social llamada \textbf{LittleSis}\cite{littlesis} que consiste en una base de datos de relaciones de \emph{quien conoce a quien} entre gente política, económica y socialmente poderosa en el mundo de las organizaciones en Estados Unidos cuyo fin es entregar el poder de la información al pueblo.\\

  El origen del nombre de \emph{LittleSis} se relaciona con el personaje \emph{Big Brother} de la novela de George Orwell llamada \emph{Nineteen Eighty-Four}. Este personaje consiste en un dictador que maneja toda la información sobre la población. \emph{Big Brother} posteriormente fue un concepto con el cual se describió un ente poderoso.\\
 
  De esta forma, al grupo compuesto por las personas poderosas y políticos de un país se les atribuye esta imagen de \emph{Big Brother}, por su poder e influencia en un país. Esta es la motivación para bautizar con el nombre \emph{LittleSis}\emph{(Little Sister)}, en oposición a \emph{Big Brother}, a una aplicación cuyo fin es entregar poder de la información a toda la población en lugar de esta minoría poderosa.\\

  La idea detrás de \emph{LittleSis} tiene la capacidad de ser replicada en muchos países con el mismo fin, esto es, el de entregar poder al pueblo e informarle sobre los posibles conflictos de interés que las autoridades locales y/o nacionales puedan tener al lidiar con el ejercicio de su poder, lo que en el caso particular de Chile, existe una iniciativa llamada \textbf{Poderopedia}\cite{poderopedia} que busca ser una réplica de \emph{LittleSis} para Chile.\\

  En el caso de \textbf{Poderopedia}, uno de los líderes de ese proyecto es el ex alumno del DCC, Álvaro Graves, quienes tienen un \textbf{modelo RDF publicado en un repositorio en Github}\cite{podervocabulary}, tienen el sitio de Poderopedia funcionando, el cual al igual que LittleSis es un repositorio centralizado de información sobre la gente de poder político y económico de Chile, sus relaciones entre sí y su relación con organizaciones.

  % -- Deficiencias de littleSis --
  Con toda sus potencialidades, tanto \emph{LittleSis} como \emph{Poderopedia}, poseen un fin muy particular. En este sentido, sólo tienen la capacidad de cubrir la información sobre las personas muy importantes del mundo político, social y económico poderoso. Pero si una persona quisiera recolectar información sobre grupos de personas (por ejemplo la red social de la comunidad investigadora en computación dentro de latinoamérica) para formar una red social, estas plataformas no sirven. Tampoco estas herramientas permiten la creación de redes sociales privadas, que cumplan los objetivos que el usuario desee, pero al mismo tiempo permitirle interactuar con los datos de otras redes sociales pertenecientes a otras personas o comunidades.\\

  % -- En esta parte se propone el proyecto ---
  El objetivo de esta memoria es llenar ese vacío, diseñando y desarrollando una herramienta para representar redes sociales, y que tengan capacidades distribuidas, donde los usuarios que poseen sus redes sociales públicas o privadas, puedan unirlas y compartirlas a voluntad.\\

  Esta herramienta posee una gran gama de aplicaciones posibles, tanto para estudios de tipo sociológico, histórico, biológico, político, etc. Un caso de ejemplo reciente es el caso de las redes de autoridades e intereses posibles dentro del ambiente educacional chileno, lo que puede dar una visión más informada sobre el entramado de intereses detrás de los problemas denunciados por el conflicto estudiantil que se vive desde el año 2011.\\

  % -- Desafíos técnicos --
  % --- Basado en el modelo de Mauro
  % --- ??
  Para lograr desarrollar esta herramienta se deben sortear una serie de desafíos técnicos, que aunque pareciera una aplicación bastante intuitiva, hasta el momento no hemos encontrado una aplicación con estas características.\\

  Entre sus mayores desafíos está el de tener una aplicación de fácil instalación y uso amigable para los usuarios. Por otro lado, con lo que respecta al modelamiento de datos, debe basarse en algún estándar flexible de representación de redes sociales que permita la interoperabilidad. Esta memoria partirá de la base y la experiencia del recientemente titulado doctor en computación Mauro San Martín, cuya tesis de doctorado consistió en diseñar un modelo para el manejo de redes sociales, de esta forma, se aprovechará el conocimiento de su experiencia de investigación, aplicándolo a un trabajo práctico y útil para la investigación y estudio de varias disciplinas.\\

  También habrá que desarrollar el aspecto de la visualización de datos, la integración con otras herramientas enfocadas al análisis de redes sociales, el modelamiento y uso de la información de redes sociales.\\

  Dado que la estructura de redes sociales está fuertemente ligada a grafos, y se necesita un estándar aceptado (para el que existan herramientas de desarrollo disponibles), se ha optado por representarlo en el modelo \textbf{RDF}. Este estándar es además es una representación que permitirá todo tipo de usos incorporándose a tecnologías de la \textbf{Web Semántica}. Finalmente, además de los anteriores, hay desafíos de negocios que se vinculan con lo técnico, como hacer algunos casos de prueba de usabilidad para entregar una herramienta de valor a usuarios y contactar diversas personas que puedan estar interesadas en la herramienta.\\

  % -- Organización del documento --
  Este informe comprende toda la experiencia al realizar esta memoria, las motivaciones, soluciones propuestas y resultados obtenidos durante el periodo de la misma. En el capítulo~\ref{chap:descripcion_proyecto} se presentan los objetivos y alcances para este trabajo. En el capítulo~\ref{chap:marco_conceptual} se abordan los conceptos relacionados a redes sociales y de desarrollo que son útiles para entender el resto del informe. En el capítulo~\ref{chap:especificacion_problema} se aborda en mayor medida el contexto en el cual se desarrolla el problema y se extraen los puntos principales a resolver. Luego en el capítulo~\ref{chap:descripcion_solucion} se documenta la arquitectura conceptual y concreta de la aplicación desarrollada para solucionar lo planteado en todo sus detalles, en el capítulo~\ref{chap:funcionamiento_solucion} se expone como esta solución funciona y finalmente se discuten las conclusiones del trabajo y sus alternativas a futuro.\\
  
\end{intro}