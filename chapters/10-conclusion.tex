\begin{conclusion}


\section{Trabajo Futuro} % (fold)
\label{sec:trabajo_futuro}

A partir de lo realizado, se pueden encontrar los siguientes pasos a futuro con este proyecto:

  \begin{enumerate}
    \item \textbf{Agregar Endpoint SPARQL}: uno de los aspectos técnicos a futuro, sería la instalación de un endpoint virtuoso, u otro para aprovechar de mejor manera la utilización del formato RDF, de esta forma, reemplazar el almacenamiento relacional de los datos por uno de grafos y lograr una mayor integración con la web semántica.
    
    \item \textbf{Mejorar Escalabilidad Aplicación}: Si el proyecto llega a un nivel popularidad grande, se puede mejorar la escalabilidad de manera tal de que la aplicación soporte redes sociales de mayor tamaño, sin perder performance de la misma.
    
    \item \textbf{Habilitar Modo Offline Aplicación}: Es posible, debido a la utilización de frameworks client side, hacer que la aplicación de edición de grafos me permita un modo offline, reemplazando el almacenamiento centralizado por uno local en el navegador, sincronizando los datos si es pertinente posteriormente.
    
    \item \textbf{Temporalidad en Redes Sociales}: se puede extender la aplicación de manera tal de que se pueda agregar temporalidad a las redes sociales, que sirva para analizar los cambios en las estructuras sociales con el paso del tiempo, aspecto que puede ser prometedor para la utilidad de esta herramienta en el estudio de ciertas disciplinas relacionadas con redes sociales.
  \end{enumerate}

% section trabajo_futuro (end)


\end{conclusion}