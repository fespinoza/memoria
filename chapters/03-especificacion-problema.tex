\chapter{Especificación del Problema}

% ### 3.2 Relevancia del problema
% 
% * el problema es relevante de tomar pues se le entrega a las personas una herramienta que les permite guardar datos de
% redes sociales en forma interactiva y gráfica.
% * permite que las personas puedan complementar su información con la información de otras personas
% * permite la generación de información compatible con la web semántica por parte de personas con conocimientos básicos
% de computación
\section{Relevancia del Problema} % (fold)
\label{sec:relevancia_del_problema}

El problema es relevante de resolver, debido a que se está entregando una herramienta que le permite a personas que no tengan conocimientos avanzados de computación, almacenar redes sociales en forma gráfica interactivamente.\\

Además estas redes sociales anteriormente creadas, se pueden complementar uniéndolas con otras redes sociales creadas por otros usuarios de la aplicación, aprovechando el conocimiento de más personas sobre estructuras sociales determinadas.\\

Otro aspecto clave de resolver este problema con este tipo de aplicación corresponde a que los datos generados por sus usuarios son compatibles con los formatos y tecnologías pertenecientes a la web semántica, por lo cual esta información puede ser relacionada con otras fuentes de conocimiento posteriormente.

% section relevancia_del_problema (end)


% ### 3.3 Usuarios Objetivo
% 
% * quienes son mis usuarios objetivo
\section{Usuarios Objetivo} % (fold)
\label{sec:usuarios_objetivo}

La herramienta que se desarrollará para esta memoria está enfocada principalmente en personas en cuya profesión requiera el estudio de diversos tipos de redes sociales, las cuales pueden encontrarse en campos como: sociología, biología, historia, negocios etc. Es decir personas con un conocimiento básico de redes sociales y el estudio de estas, aplicadas a algún campo en particular.

% section usuarios_objetivo (end)