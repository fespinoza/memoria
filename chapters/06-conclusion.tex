\chapter{Conclusión}

\section{Evaluación de la Solución} % (fold)
\label{sec:evaluacion_de_la_solucion}

% section evaluación_de_la_solución (end)

\section{Dificultades Técnicas} % (fold)
\label{sec:dificultades_tecnicas}

% ### 6.1. Dificultades encontradas
% cosas técnicas y metodológicas que fueron complejas al momento de enfrentar la memoria

Dentro del transcurso del desarrollo de esta memoria, se encontraron algunas dificultades a nivel de implementación, razón por la cual estas serán discutidas brevemente en esta sección.

\subsection{Modelamiento de Redes Sociales} % (fold)
\label{sub:modelamiento_de_redes_sociales}

En este sentido, dentro del prototipado que requirió el abordaje del desarrollo, inicialmente se pasó por alto el modelo de Mauro San Martín \cite{tesismauro}, creando un modelo básico de redes sociales a medida que las necesidades se iban presentando en términos de desarrollo, lo cual hizo que el mismo fuera más dificultoso, lo cual forzó a estudiar de mejor manera el modelo de Mauro, que mejoró tanto la estructuración de la información, además del esfuerzo requerido para desarrollar la aplicación, ahorrando dificultades por ejemplo: de tener que manejar código para actores y relaciones separadamente en vez de considerarlos un ente común llamado nodo, entre otras cosas.\\

A continuación se presenta un listado con los puntos más relevantes con las dificultades en el modelamiento y como el modelo de Mauro ayuda con esto.

  \begin{enumerate}
    \item Considerar Actores y Relaciones como Nodos, ayuda a factorizar mucha implementación debido a que su comportamiento es casi el mismo.
    
    \item La manera de como se expresan los Roles en el modelo de Mauro permite sin mayor esfuerzo implementar una relación de $M$ a $N$ dentro de un mismo modelo (Nodo), sin mayores inconvenientes e incluso provee la convención de que un Rol posee un actor y una relación, más fácilmente implementable que especificar que sólo tuviera 2 nodos.
    
    \item El modelamiento de atributos está pensado en tener diversa clase de atributos en cantidades variables para un nodo, es decir, se puede simular la funcionalidad de una base de datos sin esquema como MongoDB con una base de datos que usa esquema como MySQL.
  \end{enumerate}


% subsection modelamiento_de_redes_sociales (end)

\subsection{Interfaz de Alta Interacción en Desarrollo Web} % (fold)
\label{sub:interfaz_de_alta_interaccion_en_desarrollo_web}

Técnicamente, si se deseaba hacer una aplicación con una alta interactividad en términos de edición de grafos y que a su vez, esta tuviera las propiedades que entrega el hecho de que sea una aplicación web, el lenguaje de programación único para completar esta tarea es JavaScript, por lo tanto, en el camino, se tuvo que adoptar un enfoque distinto al desarrollo web tradicional (\emph{server side}) y optar por el desarrollo \emph{client side}, debido a que en este se pueden acceder limpiamente todos los atributos provenientes de la base de datos, tratarlos como objetos y asignarle lógica de modelos, implementar lógicas de vista mucho más complejas que lo que se puede lograr con JavaScript plano, junto con tener mejor integración con SVG, que era necesario también para este proyecto.

Al igual que en el ítem anterior, se presenta un listado con los puntos más importantes a considerar sobre las dificultades en la elección de herramientas en el desarrollo web:

  \begin{enumerate}
    \item Una aplicación que tenga mucha interacción de interfaz, con mucha lógica en estas interacciones es mucho más fácil implementarla en un framework de desarrollo del lado del cliente como EmberJS, de acuerdo a 4 demos iniciales que usaron BackboneJS, EmberJS, Ruby on Rails con enfoques distintos para comunicar los datos del modelo al Javascript.
    
    \item Un problema de EmberJS es que para el tiempo de desarrollo, otoño 2013, ember-data, el proyecto que comunica el \emph{back-end} de la aplicación con su \emph{front-end} en EmberJS, es bastante inestable, sin embargo, posee una comunidad activa que actualiza las versiones de este proyecto rápidamente.
    
    \item Para representar los elementos manipulables en el Canvas de Redes Sociales, la mejor opción fue usar D3.js para generar los gráficos vectoriales, que son tags dentro del documento HTML, por lo cual, es más simple asignarles eventos para poder interactuar con esos elementos. Además D3.js puede obtener y usar los datos de la misma forma que EmberJS los provee, por lo tanto la integración es más fácil.
  \end{enumerate}

% subsection interfaz_de_alta_interacción_en_desarrollo_web (end)


% section dificultades_técnicas (end)

\section{Trabajo Futuro} % (fold)
\label{sec:trabajo_futuro}

A partir de lo realizado, se pueden encontrar los siguientes pasos a futuro con este proyecto:

  \begin{enumerate}
    \item \textbf{Agregar Endpoint SPARQL}: uno de los aspectos técnicos a futuro, sería la instalación de un endpoint virtuoso, u otro para aprovechar de mejor manera la utilización del formato RDF, de esta forma, reemplazar el almacenamiento relacional de los datos por uno de grafos y lograr una mayor integración con la web semántica.
    
    \item \textbf{Mejorar Escalabilidad Aplicación}: Si el proyecto llega a un nivel popularidad grande, se puede mejorar la escalabilidad de manera tal de que la aplicación soporte redes sociales de mayor tamaño, sin perder performance de la misma.
    
    \item \textbf{Habilitar Modo Offline Aplicación}: Es posible, debido a la utilización de frameworks client side, hacer que la aplicación de edición de grafos me permita un modo offline, reemplazando el almacenamiento centralizado por uno local en el navegador, sincronizando los datos si es pertinente posteriormente.
    
    \item \textbf{Temporalidad en Redes Sociales}: se puede extender la aplicación de manera tal de que se pueda agregar temporalidad a las redes sociales, que sirva para analizar los cambios en las estructuras sociales con el paso del tiempo, aspecto que puede ser prometedor para la utilidad de esta herramienta en el estudio de ciertas disciplinas relacionadas con redes sociales.
    
    \item \textbf{Aspectos de Privacidad}: la aplicación se puede mejorar en términos de privacidad de los datos, explicitando que los datos agregados a la aplicación son de propiedad exclusiva de sus usuarios, que no habrá ningún tipo de mal uso de la información, además de agregar opciones para diferenciar redes privadas y públicas con sus restricciones de permisos peritnentes.
    
    % TODO aspectos legales
    \item \textbf{Aspectos Legales}:.
  \end{enumerate}

% section trabajo_futuro (end)