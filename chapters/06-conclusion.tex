\chapter{Conclusión}

\section{Evaluación de la Solución} % (fold)
\label{sec:evaluacion_de_la_solucion}

En este trabajo de memoria, aprovechando la experiencia de la memoria de Manuel Bahamonde~\cite{memoriamanuel}, se ha realizado una aplicación que permite a usuarios básicos la creación y manejo de redes sociales multi propósito y de esta forma ser una herramienta para asistir en estas tareas a investigadores de diversos campos que incluyan el estudio de redes sociales.\\

Esta aplicación se compone de un servicio de almacenamiento de redes sociales que presenta una API en formato JSON y RDF para redes sociales, que puede ser usado con otro tipo de consumidores de esta API.\\

Por otro lado una aplicación web que corre en el lado del cliente la cual cuenta con una interfaz amigable de ingreso y manipulación de datos, en donde se cumplen los objetivos de crear relaciones de aridad múltiple, agregar atributos complementarios a los nodos.\\

Además de lo anterior, los usuarios son capaces de importar o exportar en formato RDF/N3 las redes sociales que crean con la herramienta e incluso unir la información que ellos generan, funcionalidad que abre un nuevo nivel de interacción entre usuarios con respecto a la creación de diversos tipos de redes sociales.\\

Se lograron visualizaciones limpias de los datos, que pueden ser manipuladas en formatos estándares y que permiten una visión sin pérdida con distintos niveles de acercamiento o alejamiento.\\

Se representaron algunas redes sociales con la aplicación a modo de ejemplificar su utilidad, se documentó toda la experiencia en este informe que se entrega junto con el software, que posee código abierto, con las posibilidades de seguir explotando y expandiendo esta herramienta a futuro.

\subsection{Cumplimiento de Objetivos} % (fold)
\label{sub:cumplimiento_de_objetivos}

Lo expuesto anteriormente deja en evidencia como fue cumplido el objetivo general~\ref{sub:objetivo_general}, considerando los detalles que fueron observados en el resto del informe. A continuación se analizará el cumplimiento de los objetivos específicos enumerados en la sección~\ref{sub:objetivos_especificos}.

\begin{enumerate}
  \item \emph{Implementar un modelo de representación de redes sociales en RDF}. La aplicación creada posee la capacidad de exportar los datos de la red social en RDF/N3, definiendo un vocabulario simple con el cual se expresa esta información, detalles definidos en la sección~\ref{sec:exportacion_en_rdf}.
  
  \item \emph{Hacer una interfaz amigable de ingreso de datos: actores y relaciones}. La interfaz creada funciona para este ingreso de datos, que fue adaptada iterativamente para cumplir con mayor eficiencia los objetivos de ingreso de información de manera fácil para sus usuarios.
  
  \item \emph{Desarrollar un sistema centralizado que administre la asignación de identificadores únicos a los actores que los usuarios agregan a sus redes sociales}. Esto fue manejado de manera directa por el motor de base de datos utilizado, ya que este genera id's únicos para cada entidad, que al funcionar con el servidor central, asegura que las entidades creadas por distintos usuarios posean identificadores únicos.
  
  \item \emph{Unir redes sociales, combinando la información y relaciones de actores en común}. Este objetivo fue cumplido de acuerdo a lo descrito en la sección~\ref{sec:union_de_redes_sociales}.
  
  \item \emph{Visualización}
    \begin{enumerate}
      \item \emph{De la estructura general de las redes sociales creadas}. Las redes sociales poseen una representación en SVG la cual se puede manipular para la edición de estas.
      \item \emph{Específica de elementos de interés en esas redes sociales}. Este objetivo contemplaba la extracción o énfasis en subgrafos dentro del grafo principal de la red social, objetivo que no fue cumplido en este trabajo, mencionado dentro del trabajo futuro con la integración de un endpoint sparql.
    \end{enumerate}
    
  \item \emph{Representar algunas redes con el modelo implementado anteriormente, que varíen en tamaño y en complejidad}. En la sección~\ref{sec:ejemplos_de_uso} se expusieron dos casos de redes sociales modelados por medio de esta aplicación.
  
\end{enumerate}

% subsection cumplimiento_de_objetivos (end)

% section evaluación_de_la_solución (end)

\section{Dificultades Técnicas} % (fold)
\label{sec:dificultades_tecnicas}

% ### 6.1. Dificultades encontradas
% cosas técnicas y metodológicas que fueron complejas al momento de enfrentar la memoria

Dentro del transcurso del desarrollo de esta memoria, se encontraron algunas dificultades a nivel de implementación, razón por la cual éstas serán discutidas brevemente en esta sección.

\subsection{Modelamiento de Redes Sociales} % (fold)
\label{sub:modelamiento_de_redes_sociales}

En este sentido, dentro del prototipado que requirió el abordaje del desarrollo, inicialmente se pasó por alto el modelo de Mauro San Martín \cite{tesismauro}, creando un modelo básico de redes sociales a medida que las necesidades se iban presentando en términos de desarrollo. Esto hizo que el mismo fuera más dificultoso, lo cual forzó a estudiar de mejor manera el modelo de Mauro, que mejoró tanto la estructuración de la información, además del esfuerzo requerido para desarrollar la aplicación, ahorrando dificultades por ejemplo: de tener que manejar código para actores y relaciones separadamente en vez de considerarlos un ente común llamado nodo, entre otras cosas.\\

A continuación se presenta un listado con los puntos más relevantes con las dificultades en el modelamiento y como el modelo de Mauro ayuda con esto.

  \begin{enumerate}
    \item Considerar Actores y Relaciones como Nodos, ayuda a factorizar mucha implementación debido a que su comportamiento es casi el mismo.
    
    \item La manera de como se expresan los Roles en el modelo de Mauro permite sin mayor esfuerzo implementar una relación de $M$ a $N$ dentro de un mismo modelo (Nodo), sin mayores inconvenientes e incluso provee la convención de que un Rol posee un actor y una relación, más fácilmente implementable que especificar que sólo tuviera 2 nodos.
    
    \item El modelamiento de atributos está pensado en tener diversa clase de atributos en cantidades variables para un nodo, es decir, se puede simular la funcionalidad de una base de datos sin esquema como MongoDB con una base de datos que usa esquema como MySQL.
  \end{enumerate}


% subsection modelamiento_de_redes_sociales (end)

\subsection{Interfaz de Alta Interacción en Desarrollo Web} % (fold)
\label{sub:interfaz_de_alta_interaccion_en_desarrollo_web}

Técnicamente, si se deseaba hacer una aplicación con una alta interactividad en términos de edición de grafos y que a su vez, esta tuviera las propiedades que entrega el hecho de que sea una aplicación web, el lenguaje de programación único para completar esta tarea es JavaScript, por lo tanto, en el camino, se tuvo que adoptar un enfoque distinto al desarrollo web tradicional (\emph{server side}) y optar por el desarrollo \emph{client side}, debido a que en este se pueden acceder limpiamente todos los atributos provenientes de la base de datos, tratarlos como objetos y asignarle lógica de modelos, implementar lógicas de vista mucho más complejas que lo que se puede lograr con JavaScript plano, junto con tener mejor integración con SVG, que era necesario también para este proyecto.

Al igual que en el ítem anterior, se presenta un listado con los puntos más importantes a considerar sobre las dificultades en la elección de herramientas en el desarrollo web:

  \begin{enumerate}
    \item Una aplicación que tenga mucha interacción de interfaz, con mucha lógica en estas interacciones es mucho más fácil implementarla en un framework de desarrollo del lado del cliente como EmberJS, de acuerdo a 4 demos iniciales que usaron BackboneJS, EmberJS, Ruby on Rails con enfoques distintos para comunicar los datos del modelo al Javascript.
    
    \item Un problema de EmberJS es que para el tiempo de desarrollo, otoño 2013, ember-data, el proyecto que comunica el \emph{back-end} de la aplicación con su \emph{front-end} en EmberJS, es bastante inestable, sin embargo, posee una comunidad activa que actualiza las versiones de este proyecto rápidamente.
    
    \item Para representar los elementos manipulables en el Canvas de Redes Sociales, la mejor opción fue usar D3.js para generar los gráficos vectoriales, que son tags dentro del documento HTML, por lo cual, es más simple asignarles eventos para poder interactuar con esos elementos. Además D3.js puede obtener y usar los datos de la misma forma que EmberJS los provee, por lo tanto la integración es más fácil.
  \end{enumerate}

% subsection interfaz_de_alta_interacción_en_desarrollo_web (end)


% section dificultades_técnicas (end)

\section{Trabajo Futuro} % (fold)
\label{sec:trabajo_futuro}

A partir de lo realizado, los siguientes pasos a futuro pueden enriquecer este proyecto:

  \begin{enumerate}
    \item \textbf{Agregar Endpoint SPARQL}: uno de los aspectos técnicos a futuro, sería la instalación de un endpoint virtuoso, u otro para aprovechar de mejor manera la utilización del formato RDF, de esta forma, reemplazar el almacenamiento relacional de los datos por uno de grafos y lograr una mayor integración con la web semántica.
    
    \item \textbf{Mejorar Escalabilidad Aplicación}: si el proyecto llega a un nivel popularidad grande, se puede mejorar la escalabilidad de manera tal de que la aplicación soporte redes sociales de mayor tamaño, sin perder performance de la misma.
    
    \item \textbf{Agregar Funcionalidad de Clases a Familias}: esto se refiere en que se puede agregar la capacidad de agregar atributos a familias de actores y relaciones, de manera tal, que al crear un actor o relación de determinada familia con atributos, automáticamente, el nodo creado tenga estos atributos de acuerdo a la plantilla que presenta la familia.
    
    \item \textbf{Habilitar Modo Offline Aplicación}: es posible, debido a la utilización de frameworks client side, hacer que la aplicación de edición de grafos me permita un modo offline, reemplazando el almacenamiento centralizado por uno local en el navegador, sincronizando los datos si es pertinente posteriormente.
    
    \item \textbf{Temporalidad en Redes Sociales}: se puede extender la aplicación de manera tal de que se pueda agregar temporalidad a las redes sociales, que sirva para analizar los cambios en las estructuras sociales con el paso del tiempo, aspecto que puede ser prometedor para la utilidad de esta herramienta en el estudio de ciertas disciplinas relacionadas con redes sociales.
    
    \item \textbf{Aspectos de Privacidad}: la aplicación se puede mejorar en términos de privacidad de los datos, explicitando que los datos agregados a la aplicación son de propiedad exclusiva de sus usuarios, que no habrá ningún tipo de mal uso de la información, además de agregar opciones para diferenciar redes privadas y públicas con sus restricciones de permisos peritnentes.
    
    \item \textbf{Aspectos Legales}: se debe resolver además temas legales con respecto a la aplicación y la publicación de datos privados de personas, en redes sociales de tipo público, que pudieran legalmente afectar de una u otra forma a terceros por la exposición de estos datos.
    
    \item \textbf{Mejorar Interacción entre Usuarios de la Aplicación}: actualmente la forma en que 2 usuarios puede compartir redes sociales entre ellos es por medio de archivos RDF/N3 exportados desde la aplicación, sin embargo, esta experiencia puede ser mejorada haciendo una aplicación que muestre una lista de redes públicas listas para importar desde la cuenta del usuario.
    
  \end{enumerate}

% section trabajo_futuro (end)

\section{Distribución del Software y Documentación} % (fold)
\label{sec:distribucion_del_software_y_documentacion}

Todo el software, ejemplos y documentación desarrollados en este proyecto de memoria, se pueden encontrar y descargar libremente desde el sitio:\\

\begin{center}
  \url{http://fespinoza.github.io/social-graph-editor/}
\end{center}

En dicho sitio, existen instrucciones precisas de como crear un ambiente de desarrollo para la aplicación, el cual está automatizado por la herramienta de gestión de máquinas virtuales Vagrant~\cite{vagrant}, es decir, que con unos pocos comandos se puede instalar una máquina virtual que usa recetas Chef~\cite{chef} para instalar todas las aplicaciones necesarias para definir el ambiente de desarrollo las cuales están incluidas en este proyecto.\\

Además por el lado de la puesta en producción del proyecto en un servidor externo, la alternativa más fácil y rápida es Heroku, cuyos pasos de instalación se encuentran en la guía de su sitio~\cite{heroku}. En caso de querer hacer un deploy en un servidor propio, se recomienda Ubuntu 12.04 con postgresql y hacer un deployment con capistrano como una aplicación rails estándar.

% section distribucion_del_software_y_documentacion (end)