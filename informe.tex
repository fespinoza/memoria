\documentclass[upright, contnum]{umemoria}

\depto{CIENCIAS DE LA COMPUTACIÓN}
\author{FELIPE ANIBAL RICARDO ESPINOZA CASTILLO}
\title{DISEÑO Y DESARROLLO DE UNA HERRAMIENTA DE REPRESENTACIÓN Y VISUALIZACIÓN DE REDES SOCIALES CON CAPACIDADES DISTRIBUIDAS}
\auspicio{}
\date{SEPTIEMBRE 2013}
\guia{CLAUDIO GUTIÉRREZ GALLARDO}
\carrera{INGENIERO CIVIL EN COMPUTACIÓN}
\comision{GONZALO NAVARRO BADINO}{LUIS MATEU BRULE}{\ }

\setcounter{secnumdepth}{4}
\setcounter{tocdepth}{4}
\usepackage{float}

% Definiciones para usar listings
\usepackage{color}
\definecolor{gray97}{gray}{.97}
\definecolor{gray75}{gray}{.75}
\definecolor{gray45}{gray}{.45}
 
\usepackage{listings}
\lstset{
  frame=Ltb,
  framerule=0pt,
  aboveskip=0.5cm,
  framextopmargin=3pt,
  framexbottommargin=3pt,
  framexleftmargin=0.4cm,
  framesep=0pt,
  rulesep=.4pt,
  backgroundcolor=\color{gray97},
  rulesepcolor=\color{black},
  captionpos=b,
  %
  stringstyle=\ttfamily,
  showstringspaces = false,
  basicstyle=\small\ttfamily,
  commentstyle=\color{gray45},
  keywordstyle=\bfseries,
  %
  numbers=left,
  numbersep=15pt,
  numberstyle=\tiny,
  numberfirstline = false,
  breaklines=true,
}
 
% minimizar fragmentado de listados
\lstnewenvironment{listing}[1][]
   {\lstset{#1}\pagebreak[0]}{\pagebreak[0]}
 
\lstdefinestyle{rdf}
   {basicstyle=\scriptsize\bf\ttfamily}
   
\addto{\captionsspanish}{\renewcommand*{\contentsname}{Tabla de contenido}}
\addto\captionsspanish{\renewcommand\listfigurename{Índice de ilustraciones}}

\begin{document}
  \frontmatter
  \maketitle
  \begin{abstract}
  \begin{verbatim}
                                     RESUMEN DE LA MEMORIA
                                     PARA OPTAR AL TÍTULO DE:
                                     INGENIERO CIVIL EN COMPUTACIÓN
                                     POR: FELIPE ANÍBAL RICARDO ESPINOZA CASTILLO
                                     PROFESOR GUÍA: CLAUDIO GUTIÉRREZ GALLARDO
                                     FECHA: 12/07/2013
  \end{verbatim}

  \begin{center}
      \textbf{DISEÑO Y DESARROLLO DE UNA HERRAMIENTA DE REPRESENTACIÓN Y VISUALIZACIÓN DE REDES SOCIALES CON CAPACIDADES DISTRIBUIDAS}
  \end{center}
  
  % CONSIDERACIONES
  
  % - escrito en pasado
  % - motivación (por qué?)
  % - problema (qué quiero resolver?)
  % - solución desarrollada (qué cosa?)
  % - resultados obtenidos (qué gané?)
  
  % Escrito en pasado
  % Una página que resume todo
  %   - Motivación
  %   - Problema
  %   - Solución desarrollada
  %   - Resultados obtenidos
  % Muy buena ortografía y redacción

\end{abstract}
  \begin{dedicatoria}
  A María Cristina, Andrés, Jorge y Simón.
\end{dedicatoria}
  \begin{thanks}
  Tengo mucho que agradecer y a muchas personas, quienes han estado junto a mí a lo largo de toda esta etapa del proceso de aprendizaje que concluye con este trabajo de memoria. Estas personas han contribuido de alguna u otra forma en los resultados obtenidos hoy, por lo tanto es una buena instancia para agradecer por ello.
  
  En primer lugar, quiero agradecer a mi profesor guía Claudio Gutiérrez con el cual mantuve una relación de trabajo fructífera y amena, desde la idea inicial de este trabajo hasta su resultado final, siendo un apoyo fundamental para el cumplimiento de objetivos de esta, especialmente en periodos de confusión.
  
  Quiero agradecer también a las personas con las que adquirí experiencia laboral y fueron amigos desde temprana edad. Estas personas me enseñaron mucho de cómo funciona la industria del software, aspectos técnicos, de negocio, entre otros. Estas personas son: Daniel Pizarro, Hernán Sánchez, Álvaro Faúndez de \emph{Firenxis}. Rolando Abarca, Felipe Bascuñán y Juan Francisco Rodríguez de \emph{Games for Food}. David Assael, David Basulto y Gustavo García de \emph{ArchDaily}. Personas quienes me entregaron responsabilidad, confianza y conocimiento para desenvolverme como profesional desde un inicio.
  
  También tengo que agradecer a los amigos que han estado conmigo, me han entregado su cariño y me han enseñado muchas cosas útiles en la vida. Ellos son: Camilo López, Carlota González, Juan Muñoz, Paola Veliz, César Nuñez, Andrés Rebolledo, Laura Barahona, Karla Elorza, Laura Lagos, Diana Vásquez, Melanie y Melissa Villavicencio, entre muchos otros más.
  
  Además quiero mencionar a personas muy especiales que me enseñaron muchas cosas y acompañaron en las etapas que enfrenté a lo largo de los años: Giuliana Lunecke y Fabiola Quezada. Además una mención especial a Soledad Palma, quien estuvo siempre a mi lado en el desarrollo de esta memoria.
  
  Finalmente mis agradecimientos más grandes a mi familia: María Cristina, Andrés Godoy, Jorge Jesús Orellana y Simón Orellana. Quienes siempre me han entregado amor, comprensión y apoyo incondicional. De no ser por ellos no hubiera logrado las cosas que estoy logrando hoy en día.
  
  A todas estas personas, mis más sinceros agradecimientos, por todo.
\end{thanks}
  \tableofcontents
  \listoftables
  \listoffigures
  \mainmatter
  
  %=== BEGIN cuerpo de la memoria
    \begin{intro}
  % -- Contexto y littlesis --
  Existe una red social llamada \textbf{LittleSis}\cite{littlesis} que consiste en una base de datos de relaciones de \emph{quién conoce a quién} entre gente política, económica y socialmente poderosa en el mundo de las organizaciones en Estados Unidos. El fin de esa red es entregar el poder de la información a la sociedad civil.\\

  El origen del nombre de \emph{LittleSis} se relaciona con el personaje \emph{Big Brother} de la novela de George Orwell llamada \emph{Nineteen Eighty-Four}. Este personaje consiste en un dictador que maneja toda la información sobre la población. \emph{Big Brother} posteriormente fue un concepto con el cual se describió un ente poderoso.\\
 
  De esta forma, al grupo compuesto por las personas poderosas y políticos de un país se les atribuye esta imagen de \emph{Big Brother}, por su poder e influencia en un país. Esta es la motivación para bautizar con el nombre \emph{LittleSis}\emph{(Little Sister)}, en oposición a \emph{Big Brother}, a una aplicación cuyo fin es entregar poder de la información a toda la población en lugar de esta minoría poderosa.\\

  La idea detrás de \emph{LittleSis} tiene la capacidad de ser replicada en muchos países con el mismo fin, esto es, el de entregar poder a la ciudadanía e informarle sobre los posibles conflictos de interés que las autoridades locales y/o nacionales puedan tener al lidiar con el ejercicio de su poder. En el caso particular de Chile, existe una iniciativa llamada \textbf{Poderopedia}\cite{poderopedia} que busca ser una réplica de \emph{LittleSis} para Chile.\\

  En el caso de \textbf{Poderopedia}, uno de los líderes de ese proyecto es el ex alumno del DCC, Álvaro Graves, quienes tienen un \textbf{modelo RDF publicado en un repositorio en Github}\cite{podervocabulary}. Poderopedia, al igual que LittleSis, es un repositorio centralizado de información sobre la gente de poder político y económico de Chile, sus relaciones entre sí y su relación con organizaciones.

  % -- Deficiencias de littleSis --
  Con toda sus potencialidades, tanto \emph{LittleSis} como \emph{Poderopedia}, poseen un fin muy particular. En este sentido, sólo tienen la capacidad de cubrir la información sobre las personas muy importantes del mundo político, social y económico poderoso. Pero si una persona quisiera recolectar información sobre grupos de personas (por ejemplo la red social de la comunidad investigadora en computación dentro de latinoamérica) para formar una red social, estas plataformas son de poca ayuda. Tampoco estas herramientas permiten la creación de redes sociales privadas, que cumplan los objetivos que el usuario desee, pero que al mismo tiempo le permiten interactuar con los datos de otras redes sociales pertenecientes a otras personas o comunidades.\\

  % -- En esta parte se propone el proyecto ---
  El objetivo de esta memoria es llenar ese vacío, diseñando y desarrollando una herramienta para representar redes sociales, y que tenga capacidades distribuidas, donde los usuarios que poseen sus redes sociales públicas o privadas, puedan unirlas y compartirlas a voluntad.\\

  Esta herramienta posee una gran gama de aplicaciones posibles, tanto para estudios de tipo sociológico, histórico, biológico, político, etc. Un caso de ejemplo reciente es el caso de las redes de autoridades e intereses posibles dentro del ambiente educacional chileno, lo que puede dar una visión más informada sobre el entramado de intereses detrás de los problemas denunciados por el conflicto estudiantil que se vive desde el año 2011.\\

  % -- Desafíos técnicos --
  % --- Basado en el modelo de Mauro
  % --- ??
  Para lograr desarrollar esta herramienta se deben sortear una serie de desafíos técnicos. Aunque pareciera una aplicación bastante intuitiva, hasta el momento no hemos encontrado ninguna aplicación con estas características.\\

  Entre sus mayores desafíos está el de tener una aplicación de fácil instalación y uso amigable para los usuarios. Por otro lado, con lo que respecta al modelamiento de datos, debe basarse en algún estándar flexible de representación de redes sociales que permita la interoperabilidad. Esta memoria partirá de la base y la experiencia del recientemente titulado doctor en computación Mauro San Martín, cuya tesis de doctorado consistió en diseñar un modelo para el manejo de redes sociales, de esta forma, se aprovechará el conocimiento de su experiencia de investigación, aplicándolo a un trabajo práctico y útil para la investigación y estudio de varias disciplinas.\\

  También habrá que desarrollar el aspecto de la visualización de datos, la integración con otras herramientas enfocadas al análisis de redes sociales, el modelamiento y uso de la información de redes sociales.\\

  Dado que la estructura de redes sociales está fuertemente ligada a los grafos, y se necesita un estándar aceptado (para el que existan herramientas de desarrollo disponibles), se ha optado por representarlo en el modelo \textbf{RDF}. Este estándar es además una representación que permitirá todo tipo de usos incorporándose a tecnologías de la \textbf{Web Semántica}. Finalmente, además de los anteriores, hay desafíos de negocios que se vinculan con lo técnico, como hacer algunos casos de prueba de usabilidad para entregar una herramienta de valor a usuarios y contactar diversas personas que puedan estar interesadas en la herramienta.\\

  % -- Organización del documento --
  Este informe comprende la experiencia obtenida al realizar esta memoria, las motivaciones, las soluciones propuestas y los resultados obtenidos durante el periodo de la misma. En el capítulo~\ref{chap:descripcion_proyecto} se presentan los objetivos y alcances para este trabajo. En el capítulo~\ref{chap:marco_conceptual} se abordan los conceptos relacionados a redes sociales y de desarrollo que son útiles para entender el resto del informe. En el capítulo~\ref{chap:especificacion_problema} se aborda en mayor medida el contexto en el cual se desarrolla el problema y se extraen los puntos principales a resolver. Luego en el capítulo~\ref{chap:descripcion_solucion} se documenta la arquitectura conceptual y concreta de la aplicación desarrollada para solucionar lo planteado en todo sus detalles, en el capítulo~\ref{chap:funcionamiento_solucion} se expone como esta solución funciona y finalmente se discuten las conclusiones del trabajo y sus alternativas a futuro.\\
  
  La aplicación desarrollada está en \url{http://social-graph-editor.herokuapp.com}.
  
  El código fuente en \url{http://github.com/fespinoza/social-graph-editor}.
  
\end{intro}
    \chapter{Descripción del Proyecto}

\section{Objetivos} % (fold)
\label{sec:objetivos}
A continuación se presentan los objetivos para este trabajo de memoria.

\subsubsection{Objetivo General} % (fold)
\label{ssub:objetivo_general}

El objetivo de este trabajo consiste en crear una herramienta por la cual personas que estudian diversos tipos de redes
sociales como: sociólogos, periodistas, biólogos, etc; puedan representar, administrar y visualizar redes sociales
de mediana escala, contando además con la capacidad de combinar redes sociales con otros usuarios de la herramienta
para obtener redes sociales con información más completa ofreciendo esto como un servicio centralizado con la
capacidad de ser distribuido.

% subsubsection objetivo_general (end)

\subsubsection{Objetivos Específicos} % (fold)
\label{ssub:objetivos_específicos}

De lo escrito anteriormente en el objetivo general, se desprenden los siguientes objetivos intermedios:

  \begin{enumerate}
    \item Implementar un modelo de representación de redes sociales en RDF.
    \item Hacer una interfaz amigable de ingreso de datos: actores y relaciones.
    \item Desarrollar un sistema centralizado que administre la asignación de identificadores únicos a los actores que los usuarios agregan a sus redes sociales.
    \item Unir redes sociales, combinando la información y relaciones de actores en común.
    \item Visualización
      \begin{enumerate}
        \item De la estructura general de las redes sociales creadas.
        \item Específica de elementos de interés en esas redes sociales.
      \end{enumerate}
    \item Representar algunas redes con el modelo implementado anteriormente, que varíen en tamaño y en complejidad.
  \end{enumerate}

% subsubsection objetivos_específicos (end)
% section objetivos (end)

\section{Resultados Esperados} % (fold)
\label{sec:resultados_esperados}

Para llevar a cabo los objetivos expuestos en esta memoria, los resultados que se esperan consisten en: un sistema que permita el modelamiento de redes sociales en donde debe existir una aplicación que permita de manera cómoda crear estas redes sociales, complementar su información para posteriormente unir las redes de un usuario con otro en caso de que este lo quiera. Además de lo anterior, los datos generados por el sistema deben estar en un formato amigable para computadores, que faciliten la posterior interoperabilidad con otras aplicaciones y fuentes de conocimiento.

% section resultados_esperados (end)

\section{Alcances} % (fold)
\label{sec:alcances}
% Especificar algunas limitaciones al problema a resolver con el fin de que la memoria sea un paso más acotado en la
% solución final del problema:
% 
% * se resolverá el problema para redes sociales de menos de 100 actores **Tamaño pequeño, revisar este número**
% * no se considerará la posible temporalidad de la red social

Es importante mencionar, que con el objetivo de que el trabajo a realizar cumpla con las limitaciones de tiempo correspondientes a una memoria de ingeniería, el problema de la creación y manipulación de redes sociales por personas se acota en los siguientes aspectos:

\begin{itemize}
  \item El tamaño de redes sociales, se considerará pequeño con un número de alrededor 100 actores por red social, debido a que las redes serán principalmente creadas de forma manual, por tanto debe manejar un número que sea posible de alcanzar por los usuarios de la aplicación, de esta forma, ahorrando problemáticas asociadas con la escalabilidad de la aplicación para redes sociales más grandes.
  \item Dependiendo de las aplicaciones para las cuales se use el modelamiento de redes sociales, puede requerirse ver los cambios temporales que sufren estas, para analizar la estructura de una red social y su evolución en el tiempo. Dicho problema no será abordado en esta memoria.
\end{itemize}
% section alcances (end)
    \chapter{Marco Conceptual}
\label{chap:marco_conceptual}

En este capítulo se describen los conceptos necesarios para que el lector se pueda familiarizar con los conceptos usados en el resto del informe, describiendo lo que se entiende por los conceptos usados en términos de redes sociales y conceptos asociados al desarrollo de la solución.\\

En caso de así preferirlo, el lector puede saltar las secciones que estime convenientes dentro de este capítulo, en caso de dominar los conceptos y usar este capítulo como referencia en caso de necesitarlo.

\section{Conceptos de Redes Sociales} % (fold)
\label{sec:conceptos_de_redes_sociales}


A continuación se expondrán los conceptos fundamentales que se usarán en la memoria en relación a redes sociales. Estos conceptos fueron tomados de la tesis de doctorado del alumno del DCC, Mauro San Martín\cite{tesismauro}, que representan lo necesario para el entendimiento de este trabajo. Su reuso e inclusión en el informe se hace sólo para que el lector no necesite revisar la memoria de Mauro para poder ver estos conceptos.

Para construir la definición formal de redes sociales, es necesario definir algunos conceptos claves previamente, siguiendo las definiciones clásicas de este dominio\cite{sna}.

\subsection{Actor} % (fold)
\label{sub:actor}
Un actor es una entidad social, la cual está bajo estudio junto con sus interacciones sociales. En estricto rigor, los actores pueden ser definidos como individuos, corporaciones o unidades sociales colectivas. Ejemplos de actores son gente en un grupo, departamentos en una empresa o agencias de servicio público en una cuidad. El uso del término actor no significa que estas entidades necesariamente tienen la habilidad de actuar. Más allá, la mayoría de las aplicaciones en redes sociales se enfocan en colecciones de actores que son de un mismo tipo (por ejemplo, gente en un grupo de trabajo). A veces, sin embargo, la investigación necesita mirar a actores de diversos niveles o tipos conceptuales, o desde diversos conjuntos. Los datos pueden incluir atributos no relacionales asociados a los diferentes actores.
% subsection actor (end)

\subsection{Vínculos Relacionales} % (fold)
\label{sub:vinculos_relacionales}
Los actores están conectados hacia otros por vínculos sociales. El rango y tipo de estos puede ser muy amplio. La característica principal de un vínculo es que establece una conexión entre un par de actores. Algunos de los ejemplos más comunes de vínculos empleados en el análisis de redes sociales son:

  \begin{itemize}
    \item La evaluación de una persona por otra, ej: amistad declarada, gusto o respeto.
    \item Transferencia de recursos materiales, ej: transacciones de negocios, prestar o pedir prestado.
    \item Asociación o afiliación, ej: asistir conjuntamente a un evento social, o pertenecer al mismo club social.
  \end{itemize}
% subsection vinculos_relacionales (end)

\subsection{Relaciones} % (fold)
\label{sub:relaciones}
El conjunto de vínculos entre un tipo específico de miembros de un grupo se llama una relación. Por ejemplo, el conjunto de las amistades entre pares de niños en un salón de clases, o el conjunto de uniones diplomáticas entre pares de naciones en el mundo, son vínculos que definen relaciones. Para cualquier grupo de actores, podemos encontrar diversas relaciones; por ejemplo, además de las relaciones diplomáticas entre países, podemos encontrar la existencia de comercio en un determinado año. Las relaciones (o vínculos específicas) pueden tener atributos que las describen. Por ejemplo en el caso del comercio, su cantidad de transacciones total puede haber sido almacenado.
% subsection relaciones (end)

Con lo expuesto anteriormente, finalmente podemos definir una red social.

\subsection{Red Social} % (fold)
\label{sub:red_social}
Una red social consiste en uno o muchos conjuntos finitos de actores, junto con las relaciones definidas entre ellos. La presencia de información relacional es una característica crítica de las redes sociales. Una red social es un caso particular de red, de esta manera su estructura puede ser formalizada como un grafo.

En adición al uso de conceptos relacionales (Wasserman y Faust\cite{sna}) se deben tener en cuenta las siguientes consideraciones:

  \begin{itemize}
    \item Actores y sus acciones son vistos como interdependientes, más que independientes, unidades autónomas.
    \item Vínculos relacionales (conexiones) entre actores son canales de transferencia o "flujo" de recursos (material o no material).
    \item Modelos de red enfocados en individuos muestran el ambiente estructural de la red, además de proveer la capacidad de definir limitaciones en nivel individual.
    \item Los modelos de red conceptualizan estructura (social, económica, política, entre otras) como patrones generales de relaciones entre actores.
  \end{itemize}
% subsection red_social (end)


% ### 2.1 Conceptos de Redes Sociales
% 
% * explicación de que se van a reusar (copiar-pegar casi) los conceptos de redes sociales de la tesis de mauro
% 
% #### 2.1.1. Definiciones
% 
% * red social
% * nodo {NO}
% * actor
% * atributo {NO}
% * relación
% * familia {NO}
% * rol {NO}

% {NO} es pq estos conceptos son más específicos al modelo de mauro, pues se entienden mejor con el modelo

\section{Conceptos de Desarrollo} % (fold)
\label{sec:conceptos_de_desarrollo}

A continuación se explican los términos referentes al desarrollo de software del trabajo realizado en esta memoria.

\subsection{Desarrollo Web} % (fold)
\label{sub:desarrollo_web}

% * que es el desarrollo web
% * arquitectura de cliente servidor
% * que componentes típicos se encuentran en el desarrollo web
%   * servidor
%   * browser
%     * web
%     * movil
%   * servidor de bases de datos
%   * servidor de acceso a aplicaciones

El desarrollo web se denomina al desarrollo de software cuyo fin es la creación de aplicaciones que se ejecuten en computadores que puedan ser accesados vía la \emph{world-wide web}. Este estilo de desarrollo conlleva a que una aplicación que pueda estar ejecutándose en un servidor pueda entregar resultados a clientes provenientes de cualquier parte del mundo, usando una gran diversidad de dispositivos, plataformas de software, etc. Esas aplicaciones son frecuentemente accesadas por personas vía un navegador web.\\

Esta propiedad de multiplataforma de las aplicaciones desarrolladas para la web, junto con propiedades de accesibilidad desde diversos medios a las aplicaciones, ha hecho que el desarrollo web sea un enfoque de desarrollo ampliamente usado en la industria en los últimos 10 años.\\

En términos generales el desarrollo web posee una arquitectura clásica de cliente/servidor, en donde una aplicación corriendo en un servidor proporciona resultados y ejecuta instrucciones proveniente de clientes, que pueden ir desde un navegador web de p.c. o dispositivo móvil, o servir de API, \emph{Application Programming Interface}, para otras aplicaciones. A continuación se adjunta un diagrama que muestra los principales componentes cuando se habla de desarrollo web.\\

\begin{figure}[H]
  \centering
  \includegraphics[width=0.7\textwidth]{images/web_development.png}
  \caption[Componentes Presentes en el Desarrollo Web]{\emph{Componentes Presentes en el Desarrollo Web}. Descritos de manera general se tienen las siguientes capas: internet (con servidores de aplicación y servidores de base de datos), transmisión (via http en formato de documentos u otro tipo de mensajes), aplicaciones (web browser, aplicaciones desktop, aplicaciones mobile) y dispositivos (desktop pcs, notebooks, tablets y teléfonos).}
  \label{web_development}
\end{figure}

A modo de ejemplo de componentes en el desarrollo web de una aplicación que posee un servidor de ejecución de la aplicación, con un servidor de base de datos, que es accesada vía teléfono, tablet o computador por medio de un navegador web o de una interfaz API que use otra aplicación, por la cual se acceden sus datos en formato de documentos HTML (junto con sus estilos en CSS y manejo de eventos en JavaScript) o documentos en formato JSON, RDF, XML entre otros.\\

Dentro del desarrollo web, también se encuentran subcategorías que son expuestas a continuación.

\subsubsection{Desarrollo Front-End} % (fold)
\label{ssub:desarrollo_front_end}
% * es el desarrollo que compone todos los aspectos de un sistema con los cuales interactúa un cliente, como el cliente
% la arquitectura de cliente-servidor

Es el desarrollo orientado a todos los componentes con los cuales interactúa un usuario de la aplicación, enfocado en la interfaz y sus componentes gráficos, además de la experiencia usuario, la usabilidad de la aplicación, etc. Es el desarrollo del cliente que el usuario usa para acceder al core de la aplicación.\\

En términos de tecnologías, se asocia el desarrollo front-end al uso de tecnologías como: HTML, CSS, Javascript, entre otras.
% subsubsection desarrollo_front_end (end)

\subsubsection{Desarrollo Back-End} % (fold)
\label{ssub:desarrollo_back_end}

% * es desarrollo de software de la aplicación que guarda y procesa los datos del usuario

Es el desarrollo de la aplicación que almacena y posee la lógica común a todos los clientes de la aplicación, en donde el código de este tipo de desarrollo se ejecuta en el servidor. Además generalmente en el desarrollo back-end se incluye todo lo relacionado a persistencia de datos generados a través del uso de la aplicación.

% subsubsection desarrollo_back_end (end) 
% subsection desarrollo_web (end)

\subsection{Modelo MVC} % (fold)
\label{sub:modelo_mvc}
El modelo MVC (Modelo, Vista, Controlador)\cite{mvc}, consiste en un modelo para separar la lógica de una aplicación agrupándola en clases u otras unidades modulares, de acuerdo con la responsabilidad que estos módulos cumplan dentro de un sistema. A modo de ejemplo, a fin de ilustrar este concepto, se puede dar un caso de una aplicación que registre compras en un sistema de tienda online, un ejemplo de los módulos asociados a compras en MVC puede ser el siguiente:

\begin{figure}[!h]
  \centering
  \includegraphics[scale=.5]{images/mvc.png}
  \caption{Modelo MVC}
  \label{modelomvc}
\end{figure}

\begin{itemize}
  \item \textbf{Modelo}: el modelo corresponde a una clase \texttt{Compra} que contiene toda la lógica de negocio asociada a las compras, que además se asocia directamente a cómo se almacena una compra en la base de datos.
  \item \textbf{Vista}: un ejemplo de vista para una compra, puede ser una interfaz \texttt{HTML} en la cual el cliente efectúe operaciones sobre la compra, por ejemplo agregar productos, cabe destacar que la vista no ejecuta las acciones, sólo se encarga de recibirlas y enviarlas al último componente del modelo MVC, el controlador.
  \item \textbf{Controlador}: de acuerdo a lo recién expresado, el controlador es el encargado de coordinar uno o más modelos para ejecutar las acciones capturadas en la vista. En el caso de la compra, el controlador es quien efectivamente procesaría el pago de la misma.
\end{itemize}
% subsection modelo_mvc (end)

\subsection{Frameworks de Desarrollo} % (fold)
\label{sub:frameworks_de_desarollo}
Un framework de desarrollo define un marco en el cual desarrollar una aplicación. Provee una gran cantidad de funcionalidad común lista de manera de no desarrollar un proyecto desde cero, pero además de funcionalidad, provee de una estructura lógica con la cual se escribe el código, que está influida por muchas otras personas que han usado el framework en aplicaciones reales.

\subsubsection{Frameworks Server Side} % (fold)
\label{ssub:frameworks_server_side}
Dentro de los frameworks de desarrollo web existe una categoría llamada \emph{Server Side}, la cual consiste en que el código de la aplicación corre desde un servidor en internet, lo que permite que si la computación es común para muchos clientes, los resultados de esa computación pueden ser usados múltiples veces.\\

Ejemplos actuales de esta categoría de frameworks son: \texttt{Ruby on Rails}\cite{rails}, \texttt{Sinatra}\cite{sinatra} que utilizan el lenguaje \texttt{Ruby}; \texttt{Django}\cite{django}, \texttt{Pylons}\cite{pylons} para \texttt{Python}; \texttt{Spring}\cite{spring} para \texttt{Java} y \texttt{CakePHP}\cite{cake} para PHP entre otros.
% subsubsection frameworks_server_side (end)

\subsubsection{Frameworks Client Side} % (fold)
\label{ssub:frameworks_client_side}
En el último tiempo, surgió una nueva categoría de frameworks de desarrollo web llamada \emph{Client Side}, en la cual el código de la aplicación se ejecuta en el computador del usuario de la aplicación, ahorrando recursos necesarios en un servidor, además de ahorrar el tiempo de latencia entre el cómputo de una respuesta y su transmisión al equipo del usuario.\\

Comúnmente estos frameworks son escritos para ser usados con el lenguaje \texttt{Javascript}, pues posee la propiedad que todos los navegadores web implementan un motor de \texttt{Javascript} y por lo tanto el usuario no necesita instalar nada más. Ejemplos de frameworks client side son: \texttt{AngularJS}\cite{angular}, \texttt{EmberJS}\cite{ember}, \texttt{Meteor}\cite{meteor} y alternativamente \texttt{BackboneJS}\cite{backbone} que es una librería más que un framework.
% subsubsection frameworks_client_side (end)

% subsection frameworks_de_desarollo (end)

% #### 2.2.5. HTML5
% 
% * a que se llama HTML5
% * que entrega a diferencia de las versiones anteriores
% 
% ##### 2.2.5.1 SVG, Gráficos Vectoriales en la Web
% 
% * que son y de que sirven estos gráficos (redimensión, formato común xml, etc)
\subsection{HTML5} % (fold)
\label{sub:html5}

\emph{Html5}\cite{html5} es el nuevo estándar para \emph{HTML} (Hyper-Text Markup Language), definido por la WC3\cite{w3c} y Web Hypertext Application Technology Group (WHATWG), es un trabajo en progreso, sin embargo desde hace tiempo los principales navegadores soportan muchas de los nuevos elementos de HTML y sus APIs.\\

Algunas de los nuevas características más interesantes en HTML5 son: el tag \texttt{<canvas>} para dibujo en 2D, los tags \texttt{<video>} y \texttt{<audio>} para reproducción multimedia, soporte para almacenamiento local en el browser, algunos elementos específicos para el contenido como \texttt{<article>}, \texttt{<footer>}, \texttt{<header>}, \texttt{<nav>} y \texttt{<section>}; nuevos controles para formularios como para fecha, hora, email, url y búsqueda. Además de esto, soporte para SVG dentro de los sitios, característica especialmente importante para esta memoria.

\subsubsection{SVG, gráficos vectoriales en la web} % (fold)
\label{ssub:svg_graficos_vectoriales_en_la_web}

SVG, en inglés se refiere a gráficos vectoriales escalables, los cuales pueden ser usados directamente en documentos HTML5, de manera de crear gráficos complejos en un formato de tipo XML. La diferencia de este tipo de gráficos con respecto a imágenes por ejemplo, es que los gráficos SVG no pierden calidad si son redimensionados o se ven sus detalles por medio de una funcionalidad de lupa. Estos elementos son altamente animables y corresponde a una recomendación por parte de la W3C\cite{w3c}.

% subsubsection svg_gráficos_vectoriales_en_la_web (end)

% subsection html5 (end)

% section conceptos_de_desarrollo (end)

\section{Herramientas Elegidas} % (fold)
\label{sec:herramientas_elegidas}

A continuación se presentan las herramientas elegidas para el desarrollo de la aplicación y los fundamentos de estas elecciones:

\subsection{Ruby on Rails} % (fold)
\label{sub:ruby_on_rails}
Para el desarrollo del backend de la aplicación se eligió el framework de desarrollo web Ruby on Rails\cite{rails}, comúnmente conocido simplemente como Rails, en el cual se desarrolla sobre el lenguaje Ruby. Es un framework maduro con 10 años de existencia, el cual es una de las plataformas más populares para este tipo de desarrollo actualmente en Silicon Valley.\\

Dentro de las características destacables de rails se pueden destacar: sus principios de privilegiar las convenciones sobre las configuraciones, es decir, el framework entrega mucha funcionalidad hecha mientras se cumplan sus convenciones, las cuales de ser necesario se pueden sobre escribir; aplicaciones que pueden ser ejecutadas en diversos ambientes independientemente configurados como producción, testing o desarrollo.\\

Además de aspectos técnicos, dentro de las motivaciones al elegir rails, se encuentra la presencia de una gran comunidad de desarrolladores que crean muchas librerías, o en terminología ruby, 'gemas' las cuales hacen el desarrollo mucho más rápido reusando estas librerías que resuelven múltiples problemas frecuentes. Junto con lo anterior el alumno memorista posee vasta experiencia laboral en esta tecnología.
% subsection ruby_on_rails (end)

\subsection{EmberJS} % (fold)
\label{sub:emberjs}

EmberJS es un framework de desarrollo web javascript en el lado del cliente, fue creado por Yehuda Katz y Tom Dale cuando ellos hacían la segunda versión del framework web Sproutcore.\\

EmberJS se destaca debido a que se pueden escribir aplicaciones completas que son ejecutadas en el lado del cliente, con clases especializadas como \emph{modelos}, \emph{vistas}, \emph{controladores}, \emph{templates} en una versión un poco distinta del modelo MVC. Una característica principal de emberjs es la actualización automática de templates cuando la información el sistema cambia; la presencia de observadores y la capacidad de adaptar emberjs a diversos modelos de persistencia como vía API JSON o el almacenamiento local integrado en los navegadores web de última generación.\\

Para el desarrollo de la aplicación de esta memoria, debido a que se trata de una aplicación que permite editar grafos interactivamente y es vía web, es necesaria mucha integración entre el javascript de las vistas y los datos de los nodos de las redes sociales, razón por la cual fue un mejor enfoque desarrollar la aplicación con emberJS, que resultó en pruebas iniciales mucho mejor que la otra alternativa evaluada, BackboneJS\cite{backbone}.\\

Uno de los aspectos útiles de EmberJS es que uno de sus creadores, Yehuda Katz, es integrante de los equipos principales de desarrollo de Ruby on Rails y la librería de javascript jQuery. De esta forma, EmberJS está pensado para tener una muy buena integración con Rails y jQuery, potenciando su atractivo como herramienta de desarrollo en el lado del cliente.

% subsection emberjs (end)

\subsection{D3.js} % (fold)
\label{sub:d3_js}

Dentro de las herramientas principales se incluye D3.js\cite{d3}, que es una librería javascript para el despliegue y manipulación de datos como información visual vía SVG en la web. Además de lo anterior D3.js es una de las opciones más utilizadas en lo que a gráficos SVG en la web se refiere.\\

El funcionamiento básico de D3 es el siguiente: se parte por asociar un arreglo de elementos que contienen los datos con los que trabajamos y son asociados con elementos SVG, como círculos por ejemplo y para el caso del arreglo de datos, D3 tiene 3 estados: el de entrada, es decir, cuando un elemento nuevo es agregado al arreglo, D3 se encarga de crear un nuevo elemento SVG para asociarlo a este nuevo dato; el estado de actualización  es el cual que a todos los datos disponibles actualiza su posición u otras propiedades y finalmente el estado de salida es cuando un elemento del arreglo es eliminado y por consiguiente D3.js se encarga de remover su elemento SVG respectivo.

% subsection d3_js (end)

    \chapter{Especificación del Problema}

% ### 3.2 Relevancia del problema
% 
% * el problema es relevante de tomar pues se le entrega a las personas una herramienta que les permite guardar datos de
% redes sociales en forma interactiva y gráfica.
% * permite que las personas puedan complementar su información con la información de otras personas
% * permite la generación de información compatible con la web semántica por parte de personas con conocimientos básicos
% de computación
\section{Relevancia del Problema} % (fold)
\label{sec:relevancia_del_problema}

El problema es relevante de resolver, debido a que se está entregando una herramienta que le permite a personas que no tengan conocimientos avanzados de computación, almacenar redes sociales en forma gráfica interactivamente.\\

Además estas redes sociales anteriormente creadas, se pueden complementar uniéndolas con otras redes sociales creadas por otros usuarios de la aplicación, aprovechando el conocimiento de más personas sobre estructuras sociales determinadas.\\

Otro aspecto clave de resolver este problema con este tipo de aplicación corresponde a que los datos generados por sus usuarios son compatibles con los formatos y tecnologías pertenecientes a la web semántica, por lo cual esta información puede ser relacionada con otras fuentes de conocimiento posteriormente.

% section relevancia_del_problema (end)


% ### 3.3 Usuarios Objetivo
% 
% * quienes son mis usuarios objetivo
\section{Usuarios Objetivo} % (fold)
\label{sec:usuarios_objetivo}

La herramienta que se desarrollará para esta memoria está enfocada principalmente en personas en cuya profesión requiera el estudio de diversos tipos de redes sociales, las cuales pueden encontrarse en campos como: sociología, biología, historia, negocios etc. Es decir personas con un conocimiento básico de redes sociales y el estudio de estas, aplicadas a algún campo en particular.

% section usuarios_objetivo (end)
    \chapter{Descripción de la Solución}

A continuación se describe la solución propuesta con motivo de este trabajo de memoria, partiendo de conceptos más teóricos para luego abordar algunos temas de decisiones de implementación y mostrar el resultado final de la aplicación que se desarrolló.

\section{SNM, Social Network Model} % (fold)
\label{sec:snm_social_network_model}

Uno de los puntos principales de esta memoria, es poner en práctica el modelo realizado por el alumno de doctorado en el DCC, Mauro San Martín\cite{tesismauro}, quien investigó y propuso un modelo llamado \emph{Social Network Model} o SNM. Por lo tanto, la generación de redes sociales en esta aplicación tendrán en cuenta este modelo, trabajo que fue guiado por el mismo profesor guía de esta memoria, aprovechando la experiencia e investigación previa sobre el tema. Entonces, se procederá a definir y explicar brevemente en que consiste este modelo.

\subsection{Elementos de Redes Sociales} % (fold)
\label{sub:elementos_de_redes_sociales}

Para el modelo de Mauro se tienen algunos elementos pertenecientes a las redes sociales, que fueron tratados en los conceptos de redes sociales en la sección \ref{sec:conceptos_de_redes_sociales}, a los cuales se les agregan algunas restricciones:

  \begin{itemize}
    \item Los \emph{Actores}: estos tienen un identificador único y un conjunto de atributos, además pueden participar en cualquier número de relaciones.
    \item Las \emph{Relaciones} también tiene un identificador único, un conjunto de atributos y un número de actores participantes. El número de participantes puede ser uno o más, y puede cambiar sin afectar el resto de las propiedades de la relación.
    \item Los \emph{Atributos} tienen un significado asociado y un valor literal. Un atributo es identificado por el identificador del objeto al cual está añadido (acto o relación), por su significado y su valor literal. La clase de un objeto (actor o relación) es un tipo de atributo especial llamado \emph{familia}.
    \item \emph{Actores}, \emph{Relaciones}, \emph{Atributos} y sus conexiones forman una red social. Compartiendo y reusando metadata al nivel, por ejemplo, proveniencia de conjuntos de datos.
  \end{itemize}

\subsection{Definición Matemática} % (fold)
\label{sub:definicion_matematica}

Dado lo anterior, el modelo \emph{SNM} describe una red social como un grafo de la siguiente forma:

\begin{defn}
  (Red Social Generalizada) Una red social generalizada es definida como un multigrafo dirigido tripartito con etiquetas junto con una familia de funciones equitetadoras $f$ y un conjunto de familia de etiquetas $L_f$:
  
  \begin{center}
    $ G = (N, E, L_N, L_E, L_f, \iota, \nu, \epsilon, f) $
  \end{center}
  
  Donde:
  
  \begin{itemize}
    \item El conjunto de nodos $N = A \cup T \cup C$ es una unión disjunta del conjunto de actores $A$, con el conjunto de relaciones $T$ y el conjunto de atributos $C$.
    \item Existe una colección finita de familias (subconjuntos) de actores $\mathcal{A} = \{ A_1, A_2, \dotsc, A_k \}$ de manera tal que cada $A_i \subseteq A$ y $\cup_{1 \leq i \leq k}A_i = A$.
    \item Donde existe una colección finita de familias (subconjuntos) de relaciones $\mathcal{T} = \{ T_1, T_2, \dotsc, T_j \}$ donde cada $T_i \subseteq T$ y $\cup_{1 \leq i \leq k}T_i = T$.
    \item El conjunto de arcos $E = E_{AT} \cup E_{AC} \cup E_{TC}$ es la unión disjunta del conjunto de arcos entre actores y relaciones $E_{AT}$, con el conjunto de arcos entre actores y atributos $E_{AC}$, el conjunto de arcos entre atributos y relaciones $E_{TC}$.
    \item El conjunto de etiquetas de nodo $L_N = L_A \cup L_T \cup L_C$ es la unión disjunta de los conjuntos de etiquetas de actores $L_A$, de etiquetas de relaciones $L_T$ y el de etiquetas de atributos $L_C$.
    \item El conjunto de etiquetas de arcos $L_E = L_{AT} \cup L_{AC} \cup L_{TC}$ es la unión disjunta de los conjuntos de etiquetas de arcos entre actores y relaciones $L_{AT}$ (roles de participación), de etiquetas de arcos entre actores y atributos $L_{AC}$ (significado de atributos de actores) y el de etiquetas de arcos entre relaciones y atributos $L_{TC}$ (significado de atributos de relaciones).
    \item El conjunto de etiquetas de familias $L_f = L_{f_A} \cup L_{f_T}$ es la unión disjunta entre el conjunto de etiquetas de familias de actores $L_{f_A}$ y el conjunto de etiquetas de familias de relaciones $L_{f_T}$.
    \item $\iota = \{ \iota_{AT} , \iota_{AC} , \iota_{TC} \}$ es el conjunto de funciones de incidencia tales que $ \iota_{AT} : E_{AT} \longrightarrow A \times T$ es una función de incidencia que asocia cada arco de participación a un actor y su relación; $ \iota_{AC} : E_{AC} \longrightarrow A \times C$ es una función de incidencia que asocia un arco de significado a un actor y a un atributo; $ \iota_{TC} : E_{TC} \longrightarrow T \times C$ es una función de incidencia que asocia cada arco de significado a una relación y un atributo.
    \item $\nu = \{ \nu_A , \nu_T , \nu_C \}$ es un conjunto de funciones etiquetadoras de nodos tales que $ \nu_A : A \longrightarrow L_A$ es una función biyectiva desde actores a las etiquetas de actores; $ \nu_T : T \longrightarrow L_T$ es una función biyectiva desde relaciones a las etiquetas de las relaciones; $ \nu_C : C \longrightarrow L_C$ es una función biyectiva desde atributos a las etiquetas de atributos.
    \item $\epsilon = \{ \epsilon_{AT} , \epsilon_{AC} , \epsilon_{TC} \}$ es el conjunto de funciones etiquetadoras de arcos tales que $ \epsilon_{AT} : E_{AT} \longrightarrow L_{AT}$ es una función desde arcos de participación a sus etiquetas; $ \epsilon_{AC} : E_{AC} \longrightarrow L_{AC}$ y $ \epsilon_{TC} : E_{TC} \longrightarrow L_{TC}$ son funciones desde arcos de significado a sus etiquetas.
    \item $f = \{f_A, f_T\}$ es el conjunto de funciones etiquetadoras de familias tal que $f_A : \mathcal{A} \longrightarrow L_{f_A}$ es una función de familias de actores a las etiquetas de familias de actores y $f_T = \mathcal{T} \longrightarrow L_{f_T}$ es una función de familias de relaciones a las etiquetas de familias de relaciones.
    \item La siguiente condición se mantiene para todos los arcos entre el mismo par de actores y relaciones, Para todo $e_1$ y $e_2$ de manera tal que $\iota(e_1) = \iota(e_2) = (u,v)$ con $u \in A$ y $v \in T, e_1, e_2 \in E \Leftrightarrow \epsilon(e_1) \neq \epsilon(e_2)$.
    \item Cada función de etiquetado en $\nu, \epsilon$ y $f$, excepto $\nu_C$, deben ser invertibles.
    \item Para una relación $r \in T$ entre dos actores $a_1, a_2 \in A$, tal que existe $e_1, e_2 \in E$, y $\iota(e_1) = (a_1, r), \iota(e_2) = (a_2, r)$ con etiquetas $\epsilon(e_1) = p_1, \epsilon(e_2) = p_2$. La dirección de $r$ puede ser especificada por el par ordenado de las etiquetas de participación, eso es que una dirección $(p_1, p_2)$ indica que $r$ comienza en $a_1$ y termina en $a_2$, la dirección opuesta es representada por $(p_2, p_1)$.
  \end{itemize}
\end{defn}

De lo anterior se extrae alguna información sobre las características de las redes sociales y sus elementos:

\begin{itemize}
  \item Un \textbf{Nodo} puede ser un \emph{Actor}, \emph{Relación} o \emph{Atributo}.
  \item Las relaciones pueden ser de uno a múltiples actores.
  \item Los actores juegan un \textbf{Rol} en la relación.
  \item Los \emph{Actores} y \emph{Relaciones} pertenecen a \textbf{Familias de Actores} y \textbf{Familias de Relaciones} respectivamente.
  \item Una \textbf{Familia} de relación o actor, define un conjunto de actores/relaciones en común.
\end{itemize}

% subsection definición_matemática (end)

% * definición del modelo de mauro como grafo, extrayendo esto de su tesis
% 
% #### 4.1.2. Representación Gráfica
% 
% * un ejemplo de red social en la representación gráfica de la solución (extrayendo esto de su tesis)
% 
% #### 4.1.3. Características de este modelo
% 
% * un listado resumen de los principales puntos fuertes de este modelo
    \chapter{Funcionamiento de la Solución}
\label{chap:funcionamiento_solucion}

En este capítulo se mostrarán las características principales de la aplicación desarrollada y a modo de referencia para usuarios, mostrando pantallazos de la interfaz, discutiendo las decisiones de diseño de interfaces y experiencia de usuario.\\

Cuando el usuario se conecta por primera vez a la aplicación encontrará la siguiente página de inicio donde este se registrará o autentificará, junto con un video de un demo visual sobre cómo funciona la aplicación.

\begin{figure}[H]
  \centering
  \includegraphics[width=1.0\textwidth]{images/inicio.png}
  \caption[Interfaz de Inicio de La Aplicación]{\emph{Interfaz de Inicio de La Aplicación}. Esta interfaz está pensada para que el usuario se conecte o registre al sitio, junto con entregar un demo en forma de video de como esta aplicación funciona.}
  \label{inicio}
\end{figure}

A continuación se procede a mostrar los principales procesos que se llevan a cabo con la aplicación con sus respectivas interfaces de usuario vistas en detalle.

\section{Registro e Ingreso de Usuarios} % (fold)
\label{sec:registro_e_ingreso_de_usuarios}

Con el fin de mantener la privacidad de las redes sociales que sus usuarios crean, se necesita un sistema de autentificación de usuarios, el cual en este caso requiere una combinación de email y password. Al ingresar a la aplicación por primera vez, el usuario crea su cuenta, es autentificado y redirigido a la página principal donde se muestran sus redes sociales.

\begin{figure}[H]
  \centering
  \includegraphics[width=0.6\textwidth]{images/creacion_usuario.png}
  \caption{Formulario de Creación de Usuarios}
  \label{creacion_usuario}
\end{figure}

Cuando el usuario accede de nuevo a la aplicación, esta vez en lugar de registrar un usuario debe autentificarse de modo de acceder a su información vía este pequeño formulario.

\begin{figure}[H]
  \centering
  \includegraphics[width=0.7\textwidth]{images/login.png}
  \caption[Formulario de Ingreso]{\emph{Formulario de Ingreso}. Con una combinación de email y password el usuario de autentifica en el sistema.}
  \label{login}
\end{figure}

% section registro_e_ingreso_de_usuarios (end)

% ### 5.2 creación de redes sociales [screenshot]
\section{Creación de Redes Sociales} % (fold)
\label{sec:creacion_de_redes_sociales}

Dado que el usuario registró una cuenta en la aplicación, esta habilitado para crear una red social, inicialmente en la pantalla principal de redes sociales, presiona el botón \emph{Create Social Network} con el cual aparece el formulario de creación que es para darle un nombre y una descripción a la red social como detalles de esta.

\begin{figure}[H]
  \centering
  \includegraphics[width=1.0\textwidth]{images/principal_social_networks.png}
  \caption[Pantalla Principal de Redes Sociales]{\emph{Pantalla Principal de Redes Sociales}. Es la primera página luego de que el usuario se autentifica, la cual le permite crear redes sociales para luego complementar su información.}
  \label{principal_social_network}
\end{figure}

\begin{figure}[H]
  \centering
  \includegraphics[width=1.0\textwidth]{images/new_social_network.png}
  \caption{Formulario Nueva Red Social}
  \label{new_social_network}
\end{figure}

Esta información ingresada de la red social sirve como detalles de la misma y puede ser editada en cualquier momento, sin embargo, lo más relevante de la creación de la red, es la adición de contenido a la estructura de esta, lo que se cubrirá en la siguiente sección.

% section creación_de_redes_sociales (end)

\section{Edición de Redes Sociales} % (fold)
\label{sec:edicion_de_redes_sociales}

Una vez que el usuario tiene su cuenta y creó una red social, está listo para ir agregando los datos relevantes a sus redes sociales, haciendo click sobre el link con el nombre de la red social recién creada.

\begin{figure}[H]
  \centering
  \includegraphics[width=0.5\textwidth]{images/lista_redes_sociales.png}
  \caption[Lista de Redes Sociales]{\emph{Lista de Redes Sociales}. En este lugar aparecen las redes sociales creadas, donde el nombre llega a la edición del contenido de esta, además de contar en el costado con botones para editar sus detalles o eliminar la red.}
  \label{lista_redes_sociales}
\end{figure}

\subsection{Área de Edición de Redes Sociales} % (fold)
\label{sub:area_de_edicion_de_redes_sociales}

El área de edición de redes sociales consiste básicamente de 4 elementos: el canvas donde se crea el grafo de la red social (1), una barra de herramientas para la edición (2), un área donde se definen las familias de actores y relaciones (3) y un formulario con los detalles de las entidades seleccionadas (4).

\begin{figure}[H]
  \includegraphics[width=1.0\textwidth]{images/area_edicion_redes.png}
  \caption[Área de Edición de Redes]{\emph{Área de Edición de Redes}. Esta es el área principal donde el usuario completa y manipula la información de las redes sociales creadas, con números para la explicación de las diversas secciones de esta interfaz.}
  \label{area_edicion_redes}
\end{figure}

Esta interfaz está pensada para tener la menor cantidad de elementos posibles, haciendo énfasis en las herramientas primordiales necesarias para la edición del grafo.

\subsubsection{Modos de Edición} % (fold)
\label{ssub:modos_de_edicion}

Para la zona de edición se cuenta con 5 modos de edición, de acuerdo a los cuales puedo realizar acciones diversas cuando hago click dentro de la zona del canvas.

\begin{figure}[H]
  \centering
  \includegraphics[width=0.6\textwidth]{images/modos_edicion.png}
  \caption[Modos de Edición]{\emph{Modos de Edición}. Botones para seleccionar el modo de edición, con números para explicar los diversos modos.}
  \label{modos_edicion}
\end{figure}

Los modos existentes son:

  \begin{enumerate}
    \item \textbf{Herramienta de Movimiento}: esta herramienta me permite cambiar la posición de los nodos dentro del canvas a voluntad, además de al hacer click en estos seleccionarlos para la edición de sus detalles.
    \item \textbf{Modo Actor}: en este modo, al hacer click en algún punto del canvas creará un nuevo actor dentro de esta en la posición donde se especificó haciendo click. El enfoque será puesto automáticamente en el formulario de detalles del nuevo actor para rellenar rápidamente sus campos necesarios y confirmar la creación del nuevo actor.
    \item \textbf{Modo Relación}: en este modo, al hacer click en algún punto del canvas creará una nueva relación situada en esas coordenadas. El enfoque será puesto automáticamente en el formulario de edición de la relación, pero a diferencia de los actores, las relaciones son automáticamente guardadas al momento de ser creadas y cambian al \emph{modo Rol}.
    \item \textbf{Modo Rol}: en este modo, al momento de hacer click en un actor (manteniendo el botón presionado), puedo arrastrar una flecha y situarla sobre una relación, donde al soltar el botón del mouse, creará automáticamente un nuevo rol del actor seleccionado en la relación seleccionada, para después editar sus detalles en el formulario de roles correspondiente. Además, puedo repetir esta acción cuantas veces sea necesario.
    \item \textbf{Modo Unión de Nodos}: en este modo, al arrastrar un nodo (actor o relación) sobre otro del mismo tipo provocará la aparición de un cuadro de confirmación de la unión entre nodos, que de ser confirmada hará que se unan todas las propiedades de estos en uno solo, en caso contrario el nodo arrastrado retornará a su posición original.
  \end{enumerate}

% subsubsection modos_de_edición (end)

% subsection área_de_edición_de_redes_sociales (end)

\subsection{Creación y Edición de Familias} % (fold)
\label{sub:creacion_y_edicion_de_familias}

Para poder agrupar los nodos (actores y relaciones) dentro de familias, hay un área reservada para la creación propia de estas familias dentro de la red, en donde a continación se explican las principales operaciones con familias dentro de una red social.\\

\begin{figure}[H]
  \centering
  \includegraphics[width=0.6\textwidth]{images/area_familias.png}
  \caption[Área de Familias]{\emph{Área de Familias}. Detalle del área donde se crean las familias en una red social.}
  \label{area_familias}
\end{figure}

Para crear una familia se presiona el botón \emph{New} en la sección de familias, con lo que aparece un formulario con el siguiente:

\begin{figure}[H]
  \centering
  \includegraphics[width=0.7\textwidth]{images/edicion_familias.png}
  \caption[Formulario de Edición de Familias]{\emph{Formulario de Edición de Familias}. Formulario por el cual se define el nombre, color y tipo de familia.}
  \label{edicion_familias}
\end{figure}

Acá se rellena el nombre y se selecciona el color para mostrar los nodos de esta familia y el tipo de nodos a los cuales se les asignará esta familia.\\

Una vez creada, se muestra la familia en el listado con un ícono de \emph{A} para el caso de familias de actores y de \emph{R} en el caso de familias de relaciones. Donde puedo editar sus detalles o eliminarlas con los botones que se encuentran a su lado.

\begin{figure}[H]
  \centering
  \includegraphics[width=0.5\textwidth]{images/familia_creada.png}
  \caption[Detalle Familia Creada]{\emph{Detalle Familia Creada}. Un ejemplo de una familia creada, con su ícono que indica si corresponde a familia de Actores o Relaciones, además de sus botones de edición y eliminación.}
  \label{familia_creada}
\end{figure}

Es importante destacar, que se puede presionar cualquier tipo de familia, en donde al hacer esto, se cambiará al \emph{modo Actor} o \emph{modo Relación} según corresponda y a continuación cuando creo un actor o relación, por defecto pertenecerá a la familia seleccionada.

% subsection creación_y_edición_de_familias (end)

\subsection{Creación y Edición de Actores} % (fold)
\label{sub:creacion_y_edicion_de_actores}

Para crear actores, se debe definir el modo de edición a \emph{modo Actor}~\ref{ssub:modos_de_edicion}, y hacer click en el canvas donde aparecerá un actor en dicho punto y se enfocará automáticamente el formulario de creación del actor.\\

\begin{figure}[H]
  \centering
  \includegraphics[width=0.5\textwidth]{images/creacion_actor.png}
  \caption{Formulario de Creación de Actor}
  \label{creacion_actor}
\end{figure}

En este formulario se puede rellenar el nombre (opcionalmente), seleccionar las familias a las que pertenece el actor para finalmente confirmar la creación. Luego de esto, la información visual del actor es actualizada, mostrando al actor del color de la(s) familia(s) a la cual pertenece, además de un borde indicando de que es dicho actor el que está seleccionado en este momento.\\

\begin{figure}[H]
  \centering
  \includegraphics[width=0.6\textwidth]{images/ejemplo_actor_2_familias.png}
  \caption[Ejemplo Actor con 2 Familias]{\emph{Ejemplo Actor con 2 Familias}. Un actor con 2 familias tendrá un círculo con 2 colores de sus familias correspondientes.}
  \label{ejemplo_actor_2_familias}
\end{figure}

El actor puede ser editado en cualquier momento vía el formulario de actor, luego se presiona el botón de actualizar para persistir los cambios, o puede ser eliminado con el botón de borrar, después de confirmar en el cuadro de diálogo 
que aparece.

\begin{figure}[H]
  \centering
  \includegraphics[width=0.5\textwidth]{images/dialogo_eliminacion_actor.png}
  \caption{Diálogo de Eliminación de Actor}
  \label{dialogo_eliminacion_actor}
\end{figure}

% subsection creación_y_edición_de_actores (end)

\subsection{Creación y Edición de Relaciones} % (fold)
\label{sub:creacion_y_edicion_de_relaciones}

Para crear una relación, se debe seleccionar el \emph{modo Relación}~\ref{ssub:modos_de_edicion}, posteriormente hacer click dentro del canvas en donde aparecerá la nueva relación y se mostrará el formulario de edición de la relación. A diferencia de los actores, las relaciones son creadas automáticamente, lo cual cambiará al modo de edición de roles, para agregar los roles correspondientes a la relación sin perder el contexto.

\begin{figure}[H]
  \centering
  \includegraphics[width=0.5\textwidth]{images/creacion_relacion.png}
  \caption{Formulario de Creación de Relación.}
  \label{creacion_relacion}
\end{figure}

Las relaciones pueden o no tener un nombre, además de pertenecer a una familia, lo que generalmente denota el tipo de relación con la cual se está trabajando, ej: estudiaEn, dueñoDe, etc.\\

La relación puede ser editada en cualquier momento vía el formulario y presionando el botón de actualizar, o eliminada con el botón de borrar, luego de confirmar el cuadro de dialogo que aparece.

% subsection creación_y_edición_de_relaciones (end)

\subsection{Creación y Edición de Atributos en Nodos} % (fold)
\label{sub:creacion_y_edicion_de_atributos_en_nodos}

Una vez teniendo actores y relaciones creados dentro de la red social, es posible agregarles todos los atributos que se estimen conveniente por medio del formulario de edición de actores o relaciones, para esto, en la subsección de atributos en dicho formulario se puede agregar uno presionando el botón \emph{Add}, en donde puedo ingresar un atributo como un par key-value, por ejemplo puedo agregar el atributo \emph{Edad} (key) con el valor \emph{24} a un actor.

\begin{figure}[H]
  \centering
  \includegraphics[width=0.5\textwidth]{images/insercion_atributos.png}
  \caption[Añadiendo Atributos a un Actor]{\emph{Añadiendo Atributos a un Actor}. Se presiona el botón Add y luego se puede agregar el nuevo atributo como un par Key Value.}
  \label{insercion_atributos}
\end{figure}

Para editar atributos, se pueden editar directamente en sus campos y luego presionar el botón \emph{Update} para que los cambios sean persistentes, o borrar un atributo presionando el ícono junto a la definición del mismo.

% subsection creación_y_edición_de_atributos_en_nodos (end)

\subsection{Creación y Edición de Roles} % (fold)
\label{sub:creacion_y_edicion_de_roles}

Los roles representan la participación de un actor en una relación, dicha participación o rol, puede tener un nombre o no. Un ejemplo del último caso: en una relación de amistad entre dos personas, puede haber un tercer actor que fue quien los presentó, pero nuevamente, el nombre de un rol es opcional. Para crear un rol, se debe seleccionar el modo de edición de roles~\ref{ssub:modos_de_edicion}, luego con el mouse, pincho un actor y arrastro el mouse hacia una relación, al soltar el mouse el rol va a ser creado inmediatamente y el foco va a ser puesto dentro del formulario de edición del rol.

\begin{figure}[H]
  \centering
  \includegraphics[width=0.5\textwidth]{images/edicion_rol.png}
  \caption[Formulario de Edición de Rol]{\emph{Formulario de Edición de Rol}. En caso de ser necesario, se puede agregar un nombre al rol o eliminarlo desde aquí.}
  \label{edicion_rol}
\end{figure}

En este formulario puedo actualizar el nombre del rol o de ser necesario eliminar el rol. Es importante mencionar que los roles sólo serán creados desde un \emph{Actor} hacia una \emph{Relación}, cualquier otra combinación no resultará en la creación de un rol.

% subsection creación_y_edición_de_roles (end)

% section edición_de_redes_sociales (end)

\section{Exportación en RDF} % (fold)
\label{sec:exportacion_en_rdf}

Como primer paso en la integración de la aplicación con la web semántica, esta permite una exportación de las redes sociales modeladas en formato RDF/N3. Con este fin, se usa la estructura de triples descrita en el modelo de Mauro en la subsección~\ref{sub:representacion_como_triples}. Entonces, para exportar la información de la red social en este formato puede hacerse vía el botón \emph{Export in RDF/N3} en el menú de edición de redes sociales.\\

\begin{figure}[H]
  \centering
  \includegraphics[width=0.4\textwidth]{images/export_button.png}
  \caption[Botón de Exportación de la Red Social]{\emph{Botón de Exportación de la Red Social}. Con este botón se puede exportar en formato RDF/N3: la estructura de la red social, la estructura más la información visual de la red o el vocabulario de esta red.}
  \label{export_button}
\end{figure}

Por ejemplo, podemos tener el caso de una red social como la de la figura~\ref{mini_red_ejemplo}:

\begin{figure}[H]
  \centering
  \includegraphics[width=0.8\textwidth]{images/mini_red_ejemplo.png}
  \caption[Micro Ejemplo de Red Social]{\emph{Micro Ejemplo de Red Social}. Una red social que muestra la relación de un estudiante con su profesor guía.}
  \label{mini_red_ejemplo}
\end{figure}


Exportando la estructura únicamente de esta red social, que contiene 2 actores y una relación, se obtendría el siguiente resultado\\

\lstinputlisting[caption=Exportación RDF red social figura~\ref{mini_red_ejemplo}, style=rdf]{extra/exportacion.n3}
\label{lst:red_n3}

Esta representación RDF/N3 está validada~\cite{validador_rdf}. Es importante mencionar que para dicha representación RDF se define una ontología propia de la aplicación, que cuenta con las definiciones comunes de la aplicación, como por ejemplo Actores, Relaciones y Roles, Atributos, etc. El vocabulario correspondiente de la red sería el siguiente:.

\lstinputlisting[caption=Vocabulario red social figura~\ref{mini_red_ejemplo}, style=rdf]{extra/vocabulary.n3}
\label{lst:vocabulario_n3}

Además en caso de ser necesario, como se menciona antes, se puede exportar agregando la información gráfica de la red social, en donde se incluyen las posiciones de los nodos en el canvas, los colores de las familias definidas, etc.

% section exportacion_en_rdf (end)

\section{Importación en RDF} % (fold)
\label{sec:importacion_en_rdf}

Junto con la exportación en formato RDF que usa la aplicación, esta tiene la funcionalidad de importar a partir de un archivo RDF de la red social como el definido en el ejemplo de código~\ref{lst:red_n3}, para crear una red social a partir de lo importado.

\begin{figure}[H]
  \centering
  \includegraphics[width=1.0\textwidth]{images/import_sn.png}
  \caption[Importación de una Red Social]{\emph{Importación de una Red Social}. Para archivos RDF/N3 exportados previamente con la aplicación.}
  \label{import_sn}
\end{figure}

Cabe destacar que el contenido N3 de la red social puede no definir las posiciones de los nodos, en caso de un archivo que fue exportado sólo con la estructura de la red, no la información visual y por lo tanto la aplicación asigna las posiciones en base al algoritmo de layout de grafos de Fruchterman-Reingold~\cite{sna}, con lo cual crea los nodos, las familias, roles e interacciones entre ellos según lo especifica el archivo usado.

% section importación_en_rdf (end)

\section{Unión de Redes Sociales} % (fold)
\label{sec:union_de_redes_sociales}

Una de las funcionalidades claves de esta aplicación, que es una mejora con respecto a la experiencia de Manuel Bahamonde~\cite{memoriamanuel}, es la unión de redes sociales, esta consiste en que los datos generados de forma manual con la aplicación pueden ser complementados con datos generados por otros usuarios de la misma.\\

Para unir redes sociales, se utiliza el botón de creación de redes sociales y se elige la opción de unir dos redes sociales, en donde aparece el formulario de unión de redes sociales~\ref{seleccion_union}, en donde inicialmente seleccionamos un archivo RDF/N3 para importar la red, luego seleccionamos a que red se va a unir esta red social importada como indica el formulario de la figura.\\

\begin{figure}[H]
  \centering
  \includegraphics[width=0.6\textwidth]{images/seleccion_union.png}
  \caption[Selección de Redes a Unir]{\emph{Selección de Redes a Unir}. Se selecciona una red social que el usuario posee y se importa una red social externa con la que se va a unir.}
  \label{seleccion_union}
\end{figure}

Luego de presionar \emph{Next} en el formulario de unión, lo siguiente es definir las equivalencias dentro de las familias de ambas redes sociales~\ref{equivalencia_familias}, en donde a cada familia de la red original, selecciono una, más de una o ninguna equivalencia con una familia de la red social importada. Una vez estas equivalencias son definidas, se procesan en el back-end para añadir todos los datos de la red social importada en la seleccionada.\\

\begin{figure}[H]
  \centering
  \includegraphics[width=0.7\textwidth]{images/equivalencia_familias.png}
  \caption[Selección de Equivalencias entre Familias]{\emph{Selección de Equivalencias entre Familias}. En esta pantalla se seleccionan que familias de la red importada para unir, son equivalentes a las familias de la red social original, en múltiples equivalencias a la misma familia original son permitidas.}
  \label{equivalencia_familias}
\end{figure}

Con respecto a las relaciones, cabe mencionar que si en una red social está la relación $A$ con familia $F_1$ y en la otra red social se encuentra la relación $B$ con familia $F_2$, al momento de determinar que $F_1 = F_2$, la relación $B$ es eliminada y todos sus roles pasan a la relación $A$ debido a que ambas relaciones se consideran equivalentes

Luego de completar las operaciones anteriores, las redes sociales estarán unidas dentro de la original, donde los nodos fueron reordenados. Lo siguiente es definir las equivalencias entre nodos, para esto se selecciona el modo de unión en el menú de edición~\ref{ssub:modos_de_edicion}, con esto, solo basta arrastrar un nodo sobre otro y confirmar el dialogo que aparece al soltar el nodo para unirlos (figura~\ref{node_join}). Al unir un par nodos se combinan las familias, los atributos y roles de estos, comportamiento el cual se decidió de esta forma debido a que un nodo puede no tener un nombre y su contexto (familias, atributos y nodos) expresan mucho mejor la entidad que representa el nodo y de esta forma visual es más sencillo encontrar la equivalencia entre estos.

\begin{figure}[H]
  \centering
  \includegraphics[width=0.7\textwidth]{images/node_join.png}
  \caption[Operación de Unión de Nodos]{\emph{Operación de Unión de Nodos}. Explicación de los 3 pasos para unir dos nodos entre sí (actores o relaciones): seleccionar el modo de unión de nodos, tomar un nodo y arrastrarlo al que se desea unir y confirmar la operación. Con estos pasos se unen los nodos.}
  \label{node_join}
\end{figure}

% section union_de_redes_sociales (end)

\section{Ejemplos de Uso} % (fold)
\label{sec:ejemplos_de_uso}

Se muestra el funcionamiento de esta solución con algunos casos de prueba de la misma. Estos casos de prueba son ejemplos concretos de lo que se puede lograr con la aplicación, utilizando información real.

\subsubsection{Grupos de Trabajo del DCC, año 2009} % (fold)
\label{ssub:grupos_de_trabajo_del_dcc_2009}

En primer lugar se modeló con la herramienta una red social ejemplo que fue hecho por Manuel Bahamonde en su memoria~\cite{memoriamanuel} como una forma de comparar los resultados visuales y la experiencia básica al crear redes sociales con la aplicación.\\

La información fue extraída en su momento desde el sitio web del departamento de ciencias de la computación de la Universidad de Chile, con la cual se obtiene la red social de la figura~\ref{grupos_de_trabajo_dcc}, que está compuesta por los integrantes del departamento y los grupos de investigación de ese momento.

\begin{figure}[H]
  \centering
  \includegraphics[width=1.0\textwidth]{images/grupos_de_trabajo_dcc.png}
  \caption{Grupos de Trabajo DCC, año 2009}
  \label{grupos_de_trabajo_dcc}
\end{figure}

% subsubsection grupos_de_trabajo_del_dcc_año_2009 (end)

% El software que se implemento ́ esta disen ̃ado para ser utilizado por investigadores que recopilan sus datos desde varias fuentes. Se asume que estas fuentes dif ́ıcilmente cuentan con toda la informacio ́n, o estas se encuentra en un formato donde no se puede automatizar el proceso de transcribir la informacio ́n desde la fuente original hasta una red social.

% section ejemplos_de_uso (end)
    \chapter{Conclusión}

\section{Dificultades Técnicas} % (fold)
\label{sec:dificultades_tecnicas}

% ### 6.1. Dificultades encontradas
% cosas técnicas y metodológicas que fueron complejas al momento de enfrentar la memoria

Dentro del transcurso del desarrollo de esta memoria, se encontraron algunas dificultades a nivel de implementación, razón por la cual estas serán discutidas brevemente en esta sección.

\subsection{Modelamiento de Redes Sociales} % (fold)
\label{sub:modelamiento_de_redes_sociales}

En este sentido, dentro del prototipado que requirió el abordaje del desarrollo, inicialmente se pasó por alto el modelo de Mauro San Martín \cite{tesismauro}, creando un modelo básico de redes sociales a medida que las necesidades se iban presentando en términos de desarrollo, lo cual hizo que el mismo fuera más dificultoso, lo cual forzó a estudiar de mejor manera el modelo de Mauro, que mejoró tanto la estructuración de la información, además del esfuerzo requerido para desarrollar la aplicación, ahorrando dificultades por ejemplo: de tener que manejar código para actores y relaciones separadamente en vez de considerarlos un ente común llamado nodo, entre otras cosas.

% subsection modelamiento_de_redes_sociales (end)

\subsection{Interfaz de Alta Interacción en Desarrollo Web} % (fold)
\label{sub:interfaz_de_alta_interaccion_en_desarrollo_web}

Técnicamente, si se deseaba hacer una aplicación con una alta interactividad en términos de edición de grafos y que a su vez, esta tuviera las propiedades que entrega el hecho de que sea una aplicación web, el lenguaje de programación único para completar esta tarea es JavaScript, por lo tanto, en el camino, se tuvo que adoptar un enfoque distinto al desarrollo web tradicional (\emph{server side}) y optar por el desarrollo \emph{client side}, debido a que en este se pueden acceder limpiamente todos los atributos provenientes de la base de datos, tratarlos como objetos y asignarle lógica de modelos, implementar lógicas de vista mucho más complejas que lo que se puede lograr con JavaScript plano, junto con tener mejor integración con SVG, que era necesario también para este proyecto.

% subsection interfaz_de_alta_interacción_en_desarrollo_web (end)


% section dificultades_técnicas (end)

\section{Trabajo Futuro} % (fold)
\label{sec:trabajo_futuro}

A partir de lo realizado, se pueden encontrar los siguientes pasos a futuro con este proyecto:

  \begin{enumerate}
    \item \textbf{Agregar Endpoint SPARQL}: uno de los aspectos técnicos a futuro, sería la instalación de un endpoint virtuoso, u otro para aprovechar de mejor manera la utilización del formato RDF, de esta forma, reemplazar el almacenamiento relacional de los datos por uno de grafos y lograr una mayor integración con la web semántica.
    
    \item \textbf{Mejorar Escalabilidad Aplicación}: Si el proyecto llega a un nivel popularidad grande, se puede mejorar la escalabilidad de manera tal de que la aplicación soporte redes sociales de mayor tamaño, sin perder performance de la misma.
    
    \item \textbf{Habilitar Modo Offline Aplicación}: Es posible, debido a la utilización de frameworks client side, hacer que la aplicación de edición de grafos me permita un modo offline, reemplazando el almacenamiento centralizado por uno local en el navegador, sincronizando los datos si es pertinente posteriormente.
    
    \item \textbf{Temporalidad en Redes Sociales}: se puede extender la aplicación de manera tal de que se pueda agregar temporalidad a las redes sociales, que sirva para analizar los cambios en las estructuras sociales con el paso del tiempo, aspecto que puede ser prometedor para la utilidad de esta herramienta en el estudio de ciertas disciplinas relacionadas con redes sociales.
  \end{enumerate}

% section trabajo_futuro (end)
  %=== END cuerpo memoria
  
  \nocite{*}
  \bibliographystyle{plain}
  \bibliography{endings/02-bibliography}
\end{document}