\begin{abstract}
  \begin{verbatim}
                                     RESUMEN DE LA MEMORIA
                                     PARA OPTAR AL TÍTULO DE:
                                     INGENIERO CIVIL EN COMPUTACIÓN
                                     POR: FELIPE ANÍBAL RICARDO ESPINOZA CASTILLO
                                     PROFESOR GUÍA: CLAUDIO GUTIÉRREZ GALLARDO
                                     FECHA: 12/07/2013
  \end{verbatim}

  \begin{center}
      \textbf{DISEÑO Y DESARROLLO DE UNA HERRAMIENTA DE REPRESENTACIÓN Y VISUALIZACIÓN DE REDES SOCIALES CON CAPACIDADES DISTRIBUIDAS}
  \end{center}
  
  % CONSIDERACIONES
  
  % Escrito en pasado
  % Una página que resume todo
  %   - Motivación
  %   - Problema
  %   - Solución desarrollada
  %   - Resultados obtenidos
  % Muy buena ortografía y redacción
  
  % - motivación (por qué?)
  % lo que veo en littlesis y la información al pueblo
  % herramienta de apoyo a investigadores
  
  % - problema (qué quiero resolver?)
  % entregar un sistema a los usuarios para que ellos puedan fácilmente modelar la información de redes sociales
  % modelo flexible y completo para estructuras de redes sociales
  
  % - solución desarrollada (qué cosa?)
  % aplicación web del lado del cliente con un servidor
  % uso de la experiencia de Manuel Bahamonde
  % uso de la experiencia y el modelod e Mauro
  
  % - resultados obtenidos (qué gané?)
  % Una aplicación cuyo ambiente puede ser fácilmente replicable
  % Permite la colaboración entre usuarios
  % Una aplicación de modelamiento de redes sociales genéricas
  
  La existencia y utilidad de bases de datos centralizadas de relaciones de \emph{quién conoce a quién} como LittleSis y Poderopedia, que funcionan como entes informadores de los posibles conflictos de intereses entre las personas poderosas de un país y su utilidad como medios de entrega de información a las personas, que a su vez presentan un modelo que puede ser replicado en otras áreas del conocimiento.\\
  
  Con el objetivo de apoyar el trabajo de investigadores que manejan redes sociales en el estudio de sus respectivas disciplinas, que pueden no necesariamente tener conocimientos avanzados de computación, en este trabajo de memoria se abordó la creación de una herramienta que le permitiera a sus usuarios exponer información de las estructuras sociales en forma de redes, en los contextos que los ellos estimen convenientes.\\
  
  Este trabajo partió de la experiencia de una memoria similar realizada anteriormente por el ex alumno del departamento Manuel Bahamonde, mejorando su propuesta inicial, aplicando su experiencia y el modelo propuesto por el doctor en computación del departamento Mauro San Martín.\\
  
  Con lo anterior, se creó una aplicación web que le permite al usuario realizar las tareas señaladas, con una interfaz amigable y usando las últimas tecnologías en lo que a desarrollo web se refiere, que además, permite interacción entre los usuarios de la aplicación ya que pueden combinar la información que producen entre ellos.\\
  
  En ese sentido, se logró el objetivo propuesto ya que la aplicación aprovecha la flexibilidad del modelo de Mauro para modelar redes sociales de escala pequeña con una estructura que permite una adición flexible de información, además se logró un primer acercamiento a la integración con tecnologías de la web semántica, debido a que la información producida puede ser exportada en formato RDF.\\
  
  Finalmente se discutieron los desafíos encontrados durante el trabajo, además de indicar los principales focos de mejora y avance de este proyecto.

\end{abstract}