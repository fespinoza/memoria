\begin{abstract}
  % CONSIDERACIONES
  
  % Escrito en pasado
  % Una página que resume todo
  %   - Motivación
  %   - Problema
  %   - Solución desarrollada
  %   - Resultados obtenidos
  % Muy buena ortografía y redacción
  
  % - motivación (por qué?)
  % lo que veo en littlesis y la información al pueblo
  % herramienta de apoyo a investigadores
  
  % - problema (qué quiero resolver?)
  % entregar un sistema a los usuarios para que ellos puedan fácilmente modelar la información de redes sociales
  % modelo flexible y completo para estructuras de redes sociales
  
  % - solución desarrollada (qué cosa?)
  % aplicación web del lado del cliente con un servidor
  % uso de la experiencia de Manuel Bahamonde
  % uso de la experiencia y el modelod e Mauro
  
  % - resultados obtenidos (qué gané?)
  % Una aplicación cuyo ambiente puede ser fácilmente replicable
  % Permite la colaboración entre usuarios
  % Una aplicación de modelamiento de redes sociales genéricas
  
  Hoy existen diversas bases de datos centralizadas de relaciones de \emph{quién conoce a quién}, como \emph{LittleSis} y \emph{Poderopedia}, que contienen información sobre relaciones y posibles conflictos de intereses entre las personas poderosas de un país. Ellas son de mucha  utilidad como medios de entrega de información a las personas. Ellas se basan en un modelo de red social que puede ser replicado en otras áreas del conocimiento.\\
  
   Los investigadores y trabajadores sociales que se enfrentan con este tipo de estructuras sociales requieren aplicaciones que les faciliten su creación y manejo. Con el objetivo de apoyar el trabajo de estas personas, que no necesariamente tienen conocimientos avanzados de computación, en esta memoria se abordó la creación de una herramienta que le permitiera a sus usuarios crear y manejar información de las estructuras sociales en forma de redes en los contextos que ellos estimen convenientes.\\
  
  Este trabajo partió de la experiencia de una memoria similar realizada anteriormente por el ex alumno del DCC Manuel Bahamonde. En este trabajo se mejoró su propuesta inicial, aprovechando su experiencia y usando el modelo mejorado propuesto por Mauro San Martín en su tesis de doctorado.\\
  
  A partir de lo anterior, se creó una aplicación web que le permite al usuario realizar las tareas señaladas, con una interfaz amigable y usando las últimas tecnologías en lo que a desarrollo web se refiere. Esta aplicación  además permite la interacción entre los usuarios de la aplicación ya que les provee herramientas para combinar la información que producen entre ellos.\\
  
  Este trabajo presentó una gran gama de desafíos técnicos. Ellos refieren a la visualización manipulable de grafos, el manejo de eventos, la implementación del modelo de redes sociales, entre otros. Adicionalmente, se logró un primer acercamiento a la integración con tecnologías de la web semántica, debido a que la información producida puede ser importada y exportada en formato RDF. El resultado final fue una aplicación de código abierto, operativa y extensible que cumplió con los objetivos planteados inicialmente en la memoria.

\end{abstract}
\setcounter{page}{1}
\pagenumbering{roman}